\subsection{Einstellungen (View)}
\paragraph{Ziel}
Anzeigen und bearbeiten der Einstellungen.
\paragraph{Beschreibung}
Das Modul bietet eine graphische Oberfläche, in der momentane Einstellungen angezeigt werden und verändert werden können.
Es kommen folgende Typen von Auswahlelementen zum Einsatz:
\begin{enumerate}
    \item Schaltfläche: Durchführung einer Aktion durch darauf klicken
    \item Mehrfachauswahl: Auswahl aus mehreren Optionen
\end{enumerate}
Folgende Einstellungen können vom Benutzer getroffen werden:
\begin{enumerate}
    \item Sprache wählen (Mehrfachauswahl)
    \item Theme wählen (Mehrfachauswahl)
    \item Suchverlauf löschen (Schaltfläche)
    \item Speicherung des Suchverlaufs aktivieren/deaktivieren
    \item Impressum anzeigen (Schaltfläche)
    \item Datenschutz anzeigen (Schaltfläche)
    \item Hilfe anzeigen (Schaltfläche)
\end{enumerate}
\paragraph{Beteiligung an Funktionalen Anforderungen}
/FA50/ Einstellungen
\paragraph{Bereitgestellte Daten und Aufrufe}

\paragraph{Benötigte Daten und Aufrufe}
\begin{enumerate}
    \item Liste der unterstützen Sprachen und die aktuell ausgewählte Sprache.
    \item Liste der Themes und das aktuell ausgewählte Theme.
    \item Wert, ob Suchverlauf gespeichert wird.
    \item Methode: Sprache festlegen.
    \item Methode: Theme festlegen.
    \item Methode: Suchverlauf löschen.
    \item Methode: Speicherung des Suchverlaufs festlegen.
    \item Textanzeige: Impressum, Datenschutz, Hilfe anzeigen.
\end{enumerate}
\paragraph{Sonstiges}