\subsection{Backend}

Die Entwicklung erfolgt auf einzelnen Zweigen, die in den \verb#main#-Zweig überführt werden, sobald diese fertig ausgearbeitet sind.
Sobald der \verb#main#-Zweig aktualisiert wurde, wird die aktuelle Version des Backends gebaut und automatisch auf dem Server hochgeladen.
Dort ersetzt diese die alte Version und wird automatisch gestartet.

Jedes mal wenn ein Entwicklungszweig aktualisiert wird, wird eine neue Testumgebung aufgesetzt und die definierten Tests durchgeführt.
Diese Tests testen einzelne Komponenten in isolierten Umgebungen auf Funktionalität und dienen dazu, die Funktionalität beizubehalten.
Scheitern Tests, so kann der Entwicklungszweig nicht in den \verb#main#-Zweig überführt werden.

Die Überführung eines Entwicklungszweigs in den \verb#main#-Zweig benötigt die Zustimmung eines anderen Teammitgliedes.
Andernfalls kann nicht überführt werden.

Die HTTP-Schnittstellen werden hierbei nur mit geeigneten Testdatensätzen imitiert.