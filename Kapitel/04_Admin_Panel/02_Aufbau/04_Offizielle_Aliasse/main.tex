\subsection{Offizielle Aliasse}

Die Offizielle-Aliasse-Seite zeigt alle aktuell angenommenen und somit offiziellen Aliasse an.

\paragraph*{Layout}
Die Offizielle-Alias-Seite benutzt das Admin-Layout.

\paragraph*{Erreichbarkeit}
Die Offizielle-Alias-Seite ist unter \href{https://pse.itermori.de/admin/alias}{https://pse.itermori.de/admin/alias} erreichbar.

\paragraph*{Verwendbarkeit}
Ausschließlich authentifizierte Administratoren können diese Seite verwenden.

\paragraph*{Aufgabe}
Folgende Aufgaben werden erfüllt:

\begin{itemize}
    \item Anzeigen der offiziellen Aliasse
    \item Löschung von offiziellen Aliassen
    \item Filterung und Sortierung von angezeigten Aliassen nach 
    \begin{itemize}
        \item Alias
        \item Gebäudenummer
        \item Raum
    \end{itemize}
\end{itemize}

\paragraph*{Desktop}
\subparagraph*{Statischer Aufbau}
Die Seite ist unterteilt in zwei übereinanderliegende horizontale Boxen.
In der oberen Box steht der zentrierte Titel \dq Offizielle Aliasse \dq.
Die untere Box beinhaltet sämtliche angezeigten Aliase in tabellarischer Form. Die Spalten sind wie folgt gegeben:

\begin{itemize}
    \item Alias: Der Bezeichner des offiziellen Aliases
    \item Gebäudenummer: Das zugehörige Gebäude des offiziellen Aliases
    \item Raumnummer: Der zugehörige Raum des offiziellen Aliases, \dq - \dq{} falls kein zugehöriger Raum existiert
    \item Aktionen: Löschen
\end{itemize}

Die Spalten 1 bis 3 beinhalten ein Textfeld zur Anzeige der Informationen.
Die Spalten 1 bis 3 beinhalten ein Sortierungselement am rechten Ende.
Die Spalte 4 beinhaltet eine Box. Diese beinhaltet ein Element, dass die Aktion \dq Löschen \dq{} repräsentiert.
Jede Zeile enthält die eindeutige Identifizierung des Kartenobjektes. Diese wird nicht angezeigt.

\subparagraph*{Dynamisches Verhalten}
Bei Klick auf ein Sortierungselement wird die Sortierung der Tabelle geändert.
Ist die Tabelle absteigend oder nicht nach der zugehörigen Spalte sortiert, so wird beim Klick auf das Sortierelement die Tabelle aufsteigend nach dieser Spalte sortiert.
Ist die Tabelle bereits aufsteigend nach der zugehörigen Spalte sortiert, so wird beim Klick auf das Sortierelement die Tabelle absteigend nach dieser Spalte sortiert.

\paragraph*{Mobil}
\subparagraph*{Statischer Aufbau}
Die Seite ist in zwei übereinanderliegende horizontale Boxen aufgeteilt. Diese nehmen die volle ihnen zur Verfügung stehende Horizontale ein.
Die obere Box beinhaltet den zentrierten Titel \dq Offizielle Aliasse\dq.
Die untere Box beinhaltet die Sortierungsoptionen und die offiziellen Aliasse.
Die Sortierungsoptionen befinden sich ganz oben und ist in zwei Auswahlelemente aufgeteilt. 
Das erste Auswahlelement beinhaltet einen Eintrag für die Kategorien Alias, Gebäudenummer und Raumnummer.
Das zweite Auswahlelement beinhaltet die Einträge \dq Aufsteigend \dq{} und \dq Absteigend \dq{} oder Synonyme bzw. Abkürzungen dieser.
Die offiziellen Aliasse befinden sich unterhalb der Sortierungsoptionen.
Jeder offizielle Alias befindet sich in seinem eigenen Abschnitt, dieser ist wiederum in folgende Segmente aufgeteilt:

\begin{enumerate}
    \item Alias: Der Bezeichner des offiziellen Aliases
    \item Gebäudenummer: Das zugehörige Gebäude des offiziellen Aliases
    \item Raumnummer: Der zugehörige Raum des offiziellen Aliases, \dq - \dq{} falls kein zugehöriger Raum existiert
    \item Aktionen: Löschen
\end{enumerate}

Segment 1 befindet sich ganz oben im Alias-Abschnitt. Segmente 2 und 3 folgen untereinander unter Segment 1.
Segment 4 enthält eine Box. Diese beinhaltet ein Element, dass die Aktion \dq Löschen \dq{} repräsentiert.
Der Alias-Abschnitt selber enthält die eindeutige Identifizierung des Kartenobjektes \verb#mapID# auf dass sich der Alias-Vorschlag bezieht. Dieser wird nicht angezeigt.

\subparagraph*{Dynamisches Verhalten}
Die Auswahl der Sortierungsoptionen bestimmt die angezeigte Reihenfolge der offiziellen Aliasse.
Ist \dq Aufsteigend \dq{} ausgewählt, so werden die offiziellen Aliasse nach der ausgewählten Kategorie aufsteigend sortiert.
Ist \dq Absteigend \dq{} ausgewählt, so werden die offiziellen Aliasse nach der ausgewählten Kategorie absteigend sortiert.

\paragraph*{Allgemeines dynamisches Verhalten}
Wird über ein Aktions- oder ein Sortierungselement gehovert oder auf ein solches geklickt, so ändert sich die Hintergrundfarbe auf eine hellere Abwandlung der vorherigen Hintergrundfarbe.
Wird nicht mehr über das Element gehovert oder wurde etwas anderes angeklickt, so wird die Hintergrundfarbe auf den originalen Wert gesetzt.
Im folgenden bezeichnen \verb#alias# den Bezeichner des zugehörigen offiziellen Alias und \verb#mapID# die Kartenobjektidentifizierung des zugehörigen offiziellen Aliases.
Wird auf das Löschen-Aktionselement eines zugehörigen offiziellen Aliases geklickt, so wird die Methode \verb#remove# mit den Parametern \verb#mapID# und \verb#alias# aufgerufen. 