\newpage
\subsection{Alias-Vorschläge}\label{AP_Alias_Vorschlage}

Die Alias-Vorschläge-Seite repräsentiert die Schnittstelle für sämtliche \hyperref[AP_Aktionen_Alias_Vorschlage]{Aktionen}, die mit Alias-Vorschlägen gemacht werden können.

\paragraph*{Layout}
Die Alias-Vorschläge-Seite benutzt das \hyperref[AP_Admin_Layout]{Admin-Layout}.

\paragraph*{Erreichbarkeit}
Die Alias-Vorschläge-Seite ist unter \href{https://pse.itermori.de/admin/suggestions}{https://pse.itermori.de/admin/suggestions} erreichbar.

\paragraph*{Verwendbarkeit}
Ausschließlich authentifizierte Administratoren können diese Seite verwenden.

\paragraph*{Aufgabe}
Folgende Aufgaben werden erfüllt:

\begin{itemize}
    \item Anzeigen sämtlicher Alias-Vorschläge
    \item \hyperref[AP_Aktionen_Alias_Vorschlage_Annehmen]{Annehmen} von Alias-Vorschlägen
    \item \hyperref[AP_Aktionen_Alias_Vorschlage_Ablehnen]{Ablehnen} von Alias-Vorschlägen
    \item \hyperref[AP_Aktionen_Alias_Vorschlage_Blacklisten]{Blacklisten} von Alias-Vorschlägen
    \item Filterung und Sortierung von angezeigten Alias-Vorschlägen nach 
    \begin{itemize}
        \item Alias-Vorschlag
        \item Gebäudenummer
        \item Raum
        \item positives Feedback
        \item negatives Feedback
    \end{itemize}
\end{itemize}

\paragraph*{Statischer Aufbau}
\subparagraph*{Desktop}
Die Seite ist in zwei übereinanderliegende horizontale Boxen aufgeteilt.
Die obere Box beinhaltet den zentrierten Titel \dq Alias-Vorschläge\dq.
Die untere Box beinhaltet die Alias-Vorschläge in tabellarischer Form, die Spalten sind durch folgende Reihenfolge gegeben:

\begin{enumerate}
    \item Alias-Vorschlag: Der Bezeichner des Alias-Vorschlags
    \item Gebäudenummer: Das zugehörige Gebäude des Alias-Vorschlags
    \item Raumnummer: Der zugehörige Raum des Alias-Vorschlags, \dq - \dq{} falls kein zugehöriger Raum existiert
    \item positive Bewertungen: Die Anzahl an positiven Bewertungen zu einem Alias-Vorschlag
    \item negative Bewertungen: Die Anzahl an negativen Bewertungen zu einem Alias-Vorschlag
    \item \hyperref[AP_Aktionen_Alias_Vorschlage]{Aktionen}: \hyperref[AP_Aktionen_Alias_Vorschlage_Annehmen]{Annehmen}, \hyperref[AP_Aktionen_Alias_Vorschlage_Ablehnen]{Ablehnen}, \hyperref[AP_Aktionen_Alias_Vorschlage_Blacklisten]{Blacklisten}
\end{enumerate}

Die Spalten 1 bis 3 beinhalten ein Textfeld zur Anzeige der Informationen. \\
Die Spalte 4 ist unterteilt in zwei vertikale Boxen, im folgenden von links nach rechts mit 4.1 und 4.2 bezeichnet.
4.1 enthält die Anzahl positiver Bewertungen, 4.2 enthält ein Element zur Kennzeichnung dass 4.1 die Anzahl der positiven Bewertungen darstellt, zum Beispiel ein \dq + \dq. \\
Die Spalte 5 ist unterteilt in zwei vertikale Boxen, im folgenden von links nach rechts mit 5.1 und 5.2 bezeichnet.
5.1 enthält die Anzahl negativen Bewertungen, 5.2 enthält ein Element zur Kennzeichnung dass 5.1 die Anzahl der negativen Bewertungen darstellt, zum Beispiel ein \dq - \dq. \\
Spalte 6 ist unterteilt in drei vertikale Boxen. Diese enthalten jeweils ein Element, das eines der \hyperref[AP_Aktionen_Alias_Vorschlage]{Aktionen} repräsentiert. 

Die Spaltenüberschriften der Spalten 1-5 beinhalten ein Sortierungselement am linken Ende.

Der Abschnitt selber enthält die eindeutige Identifizierung des Kartenobjektes \verb#mapID# auf dass sich der Alias-Vorschlag bezieht. Dieser wird nicht angezeigt.

\subparagraph*{Mobil}
Die Seite ist in zwei übereinanderliegende horizontale Boxen aufgeteilt. Diese nehmen die volle ihnen zur Verfügung stehende Horizontale ein.\\
Die obere Box beinhaltet den zentrierten Titel \dq Alias-Vorschläge \dq.
Die untere Box beinhaltet die Sortierungsoptionen und die Alias-Vorschläge.\\
Die Sortierungsoptionen befinden sich ganz oben und ist in zwei Auswahlelemente aufgeteilt. 
Das erste Auswahlelement beinhaltet einen Eintrag für die Kategorien Alias-Vorschlag, Gebäudenummer, Raumnummer und Bewertungen.
Das zweite Auswahlelement beinhaltet die Einträge \dq Aufsteigend \dq{} und \dq Absteigend \dq{} oder Synonyme bzw. Abkürzungen dieser.\\
Die Alias-Vorschläge befinden sich unterhalb der Sortierungsoptionen.\\
Jeder Alias-Vorschlag befindet sich in seinem eigenen Abschnitt, dieser ist wiederum in folgende Segmente aufgeteilt:

\begin{enumerate}
    \item Alias-Vorschlag: Der Bezeichner des Alias-Vorschlags
    \item Gebäudenummer: Das zugehörige Gebäude des Alias-Vorschlags
    \item Raumnummer: Der zugehörige Raum des Alias-Vorschlags, \dq - \dq{} falls kein zugehöriger Raum existiert
    \item positive Bewertungen: Die Anzahl an positiven Bewertungen zu einem Alias-Vorschlag
    \item negative Bewertungen: Die Anzahl an negativen Bewertungen zu einem Alias-Vorschlag
    \item \hyperref[AP_Aktionen_Alias_Vorschlage]{Aktionen}: \hyperref[AP_Aktionen_Alias_Vorschlage_Annehmen]{Annehmen}, \hyperref[AP_Aktionen_Alias_Vorschlage_Ablehnen]{Ablehnen}, \hyperref[AP_Aktionen_Alias_Vorschlage_Blacklisten]{Blacklisten}
\end{enumerate}

Segment 1 befindet sich ganz oben im Alias-Vorschlag-Abschnitt. Segmente 2 und 3 folgen untereinander unter Segment 1. \\
Segment 4 ist in  zwei vertikale Boxen aufgeteilt (4.1 und 4.2).
4.1 enthält die Anzahl positiver Bewertungen, 4.2 enthält ein Element zur Kennzeichnung dass 4.1 die Anzahl der positiven Bewertungen darstellt, zum Beispiel ein \dq + \dq. \\
Segment 5 ist in  zwei vertikale Boxen aufgeteilt (5.1 und 5.2). 
5.1 enthält die Anzahl negativer Bewertungen, 5.2 enthält ein Element zur Kennzeichnung dass 5.1 die Anzahl der negativen Bewertungen darstellt, zum Beispiel ein \dq - \dq. \\
Segment 6 enthält drei vertikale Boxen. Jede Box ist genau einer \hyperref[AP_Aktionen_Alias_Vorschlage]{Aktion} zugeordnet und enthält ein Element um diese auszulösen.

Der Abschnitt selber enthält die eindeutige Identifizierung des Kartenobjektes \verb#mapID# auf dass sich der Alias-Vorschlag bezieht. Dieser wird nicht angezeigt.

\paragraph*{Dynamisches Verhalten}
\subparagraph*{Desktop}
Bei Klick auf ein Sortierungselement wird die Sortierung der Tabelle geändert.
Ist die Tabelle absteigend oder nicht nach der zugehörigen Spalte sortiert, so wird beim Klick auf das Sortierelement die Tabelle aufsteigend nach dieser Spalte sortiert.
Ist die Tabelle bereits aufsteigend nach der zugehörigen Spalte sortiert, so wird beim Klick auf das Sortierelement die Tabelle absteigend nach dieser Spalte sortiert.

\subparagraph*{Mobil}
Die Auswahl der Sortierungsoptionen bestimmt die angezeigte Reihenfolge der Alias-Vorschläge.
Ist \dq Aufsteigend \dq{} ausgewählt, so werden die Alias-Vorschläge nach der ausgewählten Kategorie aufsteigend sortiert.
Ist \dq Absteigend \dq{} ausgewählt, so werden die Alias-Vorschläge nach der ausgewählten Kategorie absteigend sortiert.

\paragraph*{Allgemeines dynamisches Verhalten}
Wird über ein Aktions- oder ein Sortierungselement gehovert oder auf ein solches geklickt, so ändert sich die Hintergrundfarbe auf eine hellere Abwandlung der vorherigen Hintergrundfarbe.
Wird nicht mehr über das Element gehovert oder wurde etwas anderes angeklickt, so wird die Hintergrundfarbe auf den originalen Wert gesetzt.

Wird auf ein Aktionselement eines zugehörigen Alias-Vorschlags geklickt, so wird die entsprechende \hyperref[AP_Aktionen_Alias_Vorschlage]{Aktion} ausgeführt.
Befindet sich ein Alias-Vorschlag in den \hyperref[AP_Changes]{Änderungen}, so wird dieser nicht angezeigt.

\paragraph*{Initialer Zustand}
Die Alias-Vorschläge sind absteigend nach dem Gebäude sortiert.
Es werden die Alias-Vorschläge mit den Standardwerten für positive und negative Bewertungen angefragt und angezeigt.
Die Filteroptionen sind entsprechend angepasst.

\subsubsection*{Bilder}
\begin{minipage}{\linewidth}
    \centering
    \begin{minipage}{.69\textwidth}
        \captionsetup[figure]{labelformat=empty}
        \inprelimg[width=\textwidth]{alias_vorschläge_desktop.png}
        \captionof{figure}{Desktop Ansicht}
        \captionsetup[figure]{labelformat=default}
    \end{minipage}
    \begin{minipage}{.3\textwidth}
        \captionsetup[figure]{labelformat=empty}
        \inprelimg[width=\textwidth]{alias_vorschläge_mobil.png}    
        \captionof{figure}{Mobile Ansicht}
        \captionsetup[figure]{labelformat=default}
    \end{minipage}
\end{minipage}