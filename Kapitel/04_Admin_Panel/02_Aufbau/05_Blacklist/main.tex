\newpage
\subsection{Blacklist}\label{AP_Blacklist}

Die Blacklist-Seite zeigt alle aktuellen Einträge der Blacklist an.

\paragraph*{Layout}
Die Blacklist-Seite benutzt das \hyperref[AP_Admin_Layout]{Admin-Layout}.

\paragraph*{Erreichbarkeit}
Die Blacklist-Seite ist unter \href{https://pse.itermori.de/admin/blacklist}{https://pse.itermori.de/admin/blacklist} erreichbar.

\paragraph*{Verwendbarkeit}
Ausschließlich authentifizierte Administratoren können diese Seite verwenden.

\paragraph*{Aufgabe}
Folgende Aufgaben werden erfüllt:

\begin{enumerate}
    \item Anzeigen der Blacklisteinträge
    \item \hyperref[AP_Aktionen_Blacklist_Loschen]{Löschen} einzelner Blacklisteinträge
    \item Filterung und Sortierung der Blacklisteinträge
\end{enumerate}

\paragraph*{Statischer Aufbau}
\subparagraph*{Desktop}
Die Seite ist unterteilt in zwei übereinanderliegende horizontale Boxen.
In der oberen Box steht der zentrierte Titel \dq Blacklist \dq.
Die untere Box enthält die angezeigten Blacklisteinträge in tabellarischer Form. Die Spalten dieser Tabelle sind wie folgt gegeben:

\begin{enumerate}
    \item Eintrag: Bezeichner des Blacklisteintrags
    \item \hyperref[AP_Aktionen_Blacklist]{Aktionen}: \hyperref[AP_Aktionen_Blacklist_Loschen]{Löschen}
\end{enumerate}

Die Spalte 1 enthält ein Textfeld zum Anzeigen des Eintrags.
Die Spalte 1 beinhaltet ein Sortierungselement am linken Ende.
Die Spalte 2 beinhaltet eine Box. Diese Box beinhaltet ein Element dass die \hyperref[AP_Aktionen_Blacklist]{Aktion} \hyperref[AP_Aktionen_Blacklist_Loschen]{Löschen} repräsentiert.

\subparagraph*{Mobil}
Die Seite ist in zwei übereinanderliegende horizontale Boxen aufgeteilt. Diese nehmen die volle ihnen zur Verfügung stehende Horizontale ein.\\
Die obere Box beinhaltet den zentrierten Titel \dq Blacklist\dq.
Die untere Box beinhaltet die Sortierungsoptionen und die Blacklisteinträge.\\
Die Sortierungsoptionen befinden sich ganz oben und beinhaltet ein Auswahlelement. 
Dieses Auswahlelement beinhaltet die Einträge \dq Aufsteigend \dq{} und \dq Absteigend \dq{} oder Synonyme bzw. Abkürzungen dieser.\\
Die Blacklisteinträge befinden sich unterhalb der Sortierungsoptionen.
Jeder Blacklisteintrag befindet sich in seinem eigenen Abschnitt, dieser ist wiederum in folgende Segmente aufgeteilt:

\begin{enumerate}
    \item Eintrag: Der Bezeichner des Blacklisteintrags
    \item \hyperref[AP_Aktionen_Blacklist]{Aktionen}: \hyperref[AP_Aktionen_Blacklist_Loschen]{Löschen}
\end{enumerate}

Segment 1 befindet sich ganz oben im Blacklisteintrag-Abschnitt und enthält ein Textfeld zum Anzeigen des Eintrags.
Segment 2 enthält eine Box. Diese beinhaltet ein Element, dass die Aktion \hyperref[AP_Aktionen_Blacklist]{Aktion} \hyperref[AP_Aktionen_Blacklist_Loschen]{Löschen} repräsentiert.

\paragraph*{Dynamisches Verhalten}
\subparagraph*{Desktop}
Bei Klick auf das Sortierungselement wird die Sortierung der Tabelle geändert.
Ist die Tabelle absteigend sortiert, so wird beim Klick auf das Sortierelement die Tabelle aufsteigend nach den Einträgen sortiert.
Ist die Tabelle aufsteigend sortiert, so wird beim Klick auf das Sortierelement die Tabelle absteigend nach den Einträgen sortiert.

\subparagraph*{Mobil}
Die Auswahl der Sortierungsoptionen bestimmt die angezeigte Reihenfolge der Blacklisteinträge.
Ist \dq Aufsteigend \dq{} ausgewählt, so werden die Blacklisteinträge alphabetisch aufsteigend sortiert.
Ist \dq Absteigend \dq{} ausgewählt, so werden die Blacklisteinträge alphabetisch absteigend sortiert.

\paragraph*{Allgemeines dynamisches Verhalten}
Wird über ein Aktions- oder ein Sortierungselement gehovert oder auf ein solches geklickt, so ändert sich die Hintergrundfarbe auf eine hellere Abwandlung der vorherigen Hintergrundfarbe. \\
Wird nicht mehr über das Element gehovert oder wurde etwas anderes angeklickt, so wird die Hintergrundfarbe auf den originalen Wert gesetzt.

Wird auf ein Aktionselement eines zugehörigen Blacklisteintrags geklickt, so wird die entsprechende \hyperref[AP_Aktionen_Blacklist]{Aktion} ausgeführt.
Befindet sich ein Blacklisteintrag in den \hyperref[AP_Changes]{Änderungen}, so wird dieser nicht angezeigt.

\paragraph*{Initialer Zustand}
Die Blacklist ist absteigend nach den Blacklisteinträgen sortiert.
Es werden alle Blacklisteinträge angefragt und angezeigt.
Die Filteroptionen sind entsprechend angepasst.

\subsubsection*{Bilder}
\begin{minipage}{\linewidth}
    \centering
    \begin{minipage}{.69\textwidth}
        \captionsetup[figure]{labelformat=empty}
        \inprelimg[width=\textwidth]{blacklist_desktop.png}
        \captionof{figure}{Desktop Ansicht}
        \captionsetup[figure]{labelformat=default}
    \end{minipage}
    \begin{minipage}{.3\textwidth}
        \captionsetup[figure]{labelformat=empty}
        \inprelimg[width=\textwidth]{blacklist_mobil.png}    
        \captionof{figure}{Mobile Ansicht}
        \captionsetup[figure]{labelformat=default}
    \end{minipage}
\end{minipage}