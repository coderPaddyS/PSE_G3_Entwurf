\section{Schichtenarchitektur}

Das Admin-Panel ist in drei separate Schichten aufgeteilt. Diese sind wie folgend:

\paragraph*{View}
Die erste Schicht ist die View. Diese wird durch die HTML-Dokumente des Admin-Panels mitsamt Styling durch CSS realisiert.
Sie ist durch Layouts und den Aufbau beschrieben.

\paragraph*{Controller}
Die zweite Schicht ist der Controller und steuert die Interaktionen. Der Controller erweitert die View um Funktionalität. 
Dieser besteht aus den Klassen \hyperref[AP_Framework]{Framework}, \hyperref[AP_AuthManager]{AuthManager} und \hyperref[AP_Backend]{Backend}. 
Diese Klassen befinden sich im Paket \verb#controller#. Framework ist hier die Fassade.

\paragraph*{Model}
Die letzte Schicht ist das Model. Dieses stellt den Zugriff auf die Daten sicher.
Das Model besteht aus den Klassen \hyperref[AP_Changes]{Changes}, \hyperref[AP_ChangeAction]{ChangeAction} und \hyperref[AP_Alias]{Alias} sowie dem Interface \hyperref[AP_Change]{Change}.
