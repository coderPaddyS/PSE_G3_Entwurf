\subsubsection{Annehmen}\label{AP_Aktionen_Alias_Vorschlage_Annehmen}

Die Aktion die ausgeführt wird, um einen Alias-Vorschlag anzunehmen.

\paragraph*{Ziel}
Der Alias-Vorschlag, der diese Aktion ausgelöst hat, wird zur Annahme markiert und vorbereitet.
Der zugehörige Alias-Vorschlag wird nicht mehr in der \hyperref[AP_Alias_Vorschlage]{Alias-Vorschläge-Seite} angezeigt.

\paragraph*{Vorbedingung}
Alias-Vorschlag ist vorhanden und wird angezeigt.

\paragraph*{Nachbedingung}
Alias-Vorschlag wird nicht mehr angezeigt. Der Alias-Vorschlag ist zur Annahme vorbereitet.

\paragraph*{Durchführung}
\begin{enumerate}
    \item Das Interface \hyperref[AP_Change]{\texttt{Change}} wird durch implementierung der Methode \hyperref[AP_Change_perform]{\texttt{Change::perform}} instanziiert. \\
          Dabei ruft die Methode \hyperref[AP_Change_perform]{\texttt{Change::perform}} dieser Instanz die Methode \\
          \hyperref[AP_Framework_acceptAliasSuggestion]{\texttt{Framework::acceptAliasSuggestion}} auf.
    \item Die erzeugte Instanz wird den \hyperref[AP_Changes]{Änderungen} durch die Methode \hyperref[AP_Framework_addChange]{\texttt{Framework::addChange}} mit folgenden Parametern übergeben: \begin{enumerate}
        \item action: Change
        \item metadata: JSON mit folgenden Werten: \begin{enumerate}
            \item Alias: String = Der Bezeichner des Alias-Vorschlags
            \item Gebäudenummer: String = Gebäudenummer des Alias-Vorschlags
            \item Raum: String = Raum des Alias-Vorschlags
        \end{enumerate}
        \item \dq Alias-Vorschläge \dq: String
    \end{enumerate}
\end{enumerate}