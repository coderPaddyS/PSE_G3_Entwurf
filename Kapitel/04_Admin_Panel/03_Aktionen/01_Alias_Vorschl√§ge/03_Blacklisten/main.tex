\newpage
\subsubsection{Blacklisten}\label{AP_Aktionen_Alias_Vorschläge_Blacklisten}

Die Aktion die ausgeführt wird, um einen Alias-Vorschlag auf die Blacklist zu setzen.

\paragraph*{Ziel}
Der Alias-Vorschlag, der diese Aktion ausgelöst hat, wird zur Setzung auf die Blacklist markiert und vorbereitet.
Der zugehörige Alias-Vorschlag wird nicht mehr in der Alias-Vorschläge-Seite angezeigt.

\paragraph*{Vorbedingung}
Alias-Vorschlag ist vorhanden und wird angezeigt.

\paragraph*{Nachbedingung}
Alias-Vorschlag wird nicht mehr angezeigt. Der Alias-Vorschlag ist zur Setzung auf die Blacklist vorbereitet.

\paragraph*{Durchführung}
\begin{enumerate}
    \item Das Interface \verb#Change# wird durch implementierung der Methode \verb#Change::perform# instanziiert. \\
          Dabei ruft die Methode \verb#Change::perform# dieser Instanz die \\ 
          Methode \verb#Backend::blacklistAliasSuggestion# mit folgenden Parametern auf \begin{enumerate}
              \item alias: String = Der Bezeichner des Alias-Vorschlags
              \item mapID: String = Der Kartenobjektidentifizierer des Alias-Vorschlags
          \end{enumerate}
    \item Die erzeugte Instanz wird den Changes durch die Methode \verb#Changes::addChange# mit folgenden Parametern übergeben: \begin{enumerate}
        \item action: Change
        \item metadata: JSON mit folgenden Werten: \begin{enumerate}
            \item Alias: String = Der Bezeichner des Alias-Vorschlags
            \item Gebäudenummer: String = Gebäudenummer des Alias-Vorschlags
            \item Raum: String = Raum des Alias-Vorschlags
        \end{enumerate}
    \end{enumerate}
\end{enumerate}