\subsubsection{Löschen}\label{AP_Aktionen_Blacklist_Loschen}

Die Aktion die ausgeführt wird, um einen Blacklisteintrag zu löschen.

\paragraph*{Ziel}
Der Blacklisteneintrag, der diese Aktion ausgelöst hat, wird zur Löschung markiert und vorbereitet.
Der zugehörige Blacklisteneintrag wird nicht mehr in der \hyperref[AP_Blacklist]{Blacklist-Seite} angezeigt.

\paragraph*{Vorbedingung}
Blacklisteintrag ist vorhanden und wird angezeigt.

\paragraph*{Nachbedingung}
Blacklisteintrag wird nicht mehr angezeigt. Der Blacklisteintrag ist zur Löschung vorbereitet.

\paragraph*{Durchführung}
\begin{enumerate}
    \item Das Interface \hyperref[AP_Change]{\texttt{Change}} wird durch implementierung der Methode \hyperref[AP_Change_perform]{\texttt{Change::perform}} instanziiert. \\
          Dabei ruft die Methode \hyperref[AP_Change_perform]{\texttt{Change::perform}} dieser Instanz die Methode \\
          \hyperref[AP_Framework_deleteBlacklistEntry]{\texttt{Framework::deleteBlacklistEntry}} auf.
    \item Die erzeugte Instanz wird den \hyperref[AP_Changes]{Änderungen} durch die Methode \hyperref[AP_Framework_addChange]{\texttt{Framework::addChange}} mit folgenden Parametern übergeben: \begin{enumerate}
        \item action: Change
        \item metadata: JSON mit folgenden Werten: \begin{enumerate}
            \item entry: String = Blacklisteintrag.
        \end{enumerate}
        \item \dq Blacklist \dq: String
    \end{enumerate}
\end{enumerate}
