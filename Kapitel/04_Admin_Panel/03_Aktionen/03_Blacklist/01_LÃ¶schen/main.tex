\subsubsection{Löschen}

Die Aktion die ausgeführt wird, um einen Blacklisteintrag zu löschen.

\paragraph*{Ziel}
Der Blacklisteneintrag, der diese Aktion ausgelöst hat, wird zur Löschung markiert und vorbereitet.
Der zugehörige Blacklisteneintrag wird nicht mehr in der Blacklist-Seite angezeigt.

\paragraph*{Vorbedingung}
Blacklisteintrag ist vorhanden und wird angezeigt.

\paragraph*{Nachbedingung}
Blacklisteintrag wird nicht mehr angezeigt. Der Blacklisteintrag ist zur Löschung vorbereitet.

\paragraph*{Durchführung}
\begin{enumerate}
    \item Das Interface \verb#Change# wird durch implementierung der \\ 
          Methode \verb#Change::perform# instanziiert.
          Dabei ruft die Methode \verb#Change::perform# dieser Instanz die Methode \verb#Backend::deleteBlacklistEntry# mit folgenden Parametern auf \begin{enumerate}
              \item entry: Der zu löschende Blacklisteintrag
          \end{enumerate}
    \item Die erzeugte Instanz wird den Changes durch die Methode \verb#Changes::addChange# mit folgenden Parametern übergeben: \begin{enumerate}
        \item action : Change
        \item metadata : JSON mit folgenden Werten: \begin{enumerate}
            \item entry : String = Blacklisteintrag
        \end{enumerate}
    \end{enumerate}
\end{enumerate}