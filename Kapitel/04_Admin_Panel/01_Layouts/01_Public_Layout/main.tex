\subsection{Public-Layout}

\subsubsection{Desktop}
\paragraph*{Statischer Aufbau}
Das Public-Layout besteht aus einer zentrierten 3x3 Tabelle.
Im folgenden sind die Felder zeilenweise durchnummeriert, beginnend mit der 1.

\subparagraph*{Seiteninhalt}
Die Felder 1, 2, 4 und 5 sind zu einem großen Feld zusammengefasst. In diesem Feld wird der Inhalt geladen,
der durch die Seite definiert wird, die das Public-Layout nutzt.

\subparagraph*{Anmeldung}
Die Felder 3 und 6 sind zusammengefasst. In diesen Feldern existiert ein Element pro Anmeldemöglichkeit.

\subparagraph*{Verlinkungen}
Die Felder 7 bis 9 sind zusammengefasst. In diesen Feldern existieren Verlinkungen zu wichtigen Ressourcen dieses Projektes.

\paragraph*{Dynamisches Verhalten}
\subparagraph*{Seiteninhalt}
Das dynamische Verhalten des Seiteninhalts wird durch die Seite definiert, die das Public-Layout verwendet.

\subparagraph*{Anmeldung}
Wird ein Anmeldungselement geklickt, so wird der Benutzer auf die Anmeldeseite des entsprechenden Dienstes weitergeleitet, um sich zu authentifizieren.
Anschließend wird er wieder zur vorherigen Seite zurückgeleitet.

\subparagraph*{Verlinkungen}
Wird eine Verlinkung geklickt, so öffnet sich das Verlinkungsziel in einem neuen Browser-Tab.

\paragraph*{Mobil}

\paragraph*{Allgemeines dynamisches Verhalten}
Angemeldete Administratoren werden direkt zum Dashboard weitergeleitet.
Ein Element, über das gehovert oder auf das geklickt wird, ändert seine Hintergrundfarbe zu einer helleren Abwandlung der eigentlichen Hintergrundfarbe.
