\subsubsection{SettingsScreen}
\paragraph*{Typ}
\texttt{Composable}.
\paragraph*{Beschreibung}
Dieses \texttt{Composable} zeigt die Einstellungen an. 
Der Benutzer kann die Einstellungen durch Interaktion mit dem SettingsScreen festlegen und ändern.\\
Der SettingsScreen besitzt einen ISettingsController, der die Daten bereitstellt und die Änderungen an den Einstellungen durchführt. 

\paragraph*{Aufbau}
Der SettingsScreen zeigt folgende Einstellungs-Möglichkeiten untereinander an:
\begin{itemize}
    \item Anmelde-Schaltfläche: Ist der Benutzer nicht angemeldet, wird angezeigt, dass er sich anmelden kann. 
    Klickt der Benutzer auf diese Schaltfläche, zeigt die App die Anmeldung an.\\
    Ist der Benutzer angemeldet, wird angezeigt, dass er angemeldet ist. Außerdem wird die UserId angezeigt. 
    Klickt er lange auf diese, wird die UserId in die Zwischenablage kopiert. 
    Es wird über einen \href{https://developer.android.com/guide/topics/ui/notifiers/toasts}{Toast} angezeigt, dass 
    die UserId in die Zwischenablage kopiert wurde.
    \item Sprach-Schaltfläche: Klickt der Benutzer auf diese Schaltfläche, bekommt er in einem Untermenü alle verfügbaren Sprachen angezeigt. 
    Wählt er davon eine aus, wird die App in der ausgewählten Sprache angezeigt. Die App bleibt im Untermenü.\\
    Im Untermenü befindet sich oben links ein Zurück-Button. Wird dieser gedrückt oder die Zurückfunktion des Endgeräts betätigt, 
    wird wieder die Gesamtansicht angezeigt.
    \item Theme-Schaltfläche: Klickt der Benutzer auf diese Schaltfläche, bekommt er in einem Untermenü alle verfügbaren Themes angezeigt. 
    Wählt er davon eines aus, wird die App im gewählten Theme angezeigt. Die App bleibt im Untermenü.\\
    Im Untermenü befindet sich oben links ein Zurück-Button. Wird dieser gedrückt oder die Zurückfunktion des Endgeräts betätigt, 
    wird wieder die Gesamtansicht angezeigt.
    \item Suchverlauf speichern: In dieser Schaltfläche befindet sich ein Schalter. 
    Am Zustand des Schalters lässt sich erkennen, ob die Speicherung des Suchverlaufs an- oder ausgeschaltet ist. 
    Durch Klicken auf den Schalter lässt sich dies wechseln.
    \item Suchverlauf löschen: Wird diese Schaltfläche angeklickt, wird der Suchverlauf gelöscht.
    \item Impressum anzeigen: Wird auf diese Schaltfläche geklickt, wechselt die App in die Dokumentenansicht und zeigt dort das Impressum der App an.
    \item Datenschutz anzeigen: Wird auf diese Schaltfläche geklickt, wechselt die App in die Dokumentenansicht und zeigt dort die Datenschutzerklärung der App an.
    \item Hilfe anzeigen: Wird auf diese Schaltfläche geklickt, wechselt die App in die Dokumentenansicht und zeigt dort eine Hilfeseite an.
    \item Lizenzen anzeigen: Wird auf diese Schaltfläche geklickt, wechselt die App in die Dokumentenansicht und zeigt dort Open-Source-Lizenzen der App an.
\end{itemize}

\subsubsection*{Bilder}
\begin{minipage}{\linewidth}
    \centering
    \begin{minipage}{.49\textwidth}
        \captionsetup[figure]{labelformat=empty}
        \inprelimg[width=\textwidth]{Einstellungen_oben.png}
        \captionof{figure}{Einstellungen (oben)}
        \captionsetup[figure]{labelformat=default}
    \end{minipage}
    \begin{minipage}{.49\textwidth}
        \captionsetup[figure]{labelformat=empty}
        \inprelimg[width=\textwidth]{Einstellungen_unten.png}
        \captionof{figure}{Einstellungen (unten)}
        \captionsetup[figure]{labelformat=default}
    \end{minipage}
\end{minipage}
\begin{minipage}{\linewidth}
    \centering
    \begin{minipage}{.49\textwidth}
        \captionsetup[figure]{labelformat=empty}
        \inprelimg[width=\textwidth]{Sprache.png}
        \captionof{figure}{Spracheinstellungen}
        \captionsetup[figure]{labelformat=default}
    \end{minipage}
    \begin{minipage}{.49\textwidth}
        \captionsetup[figure]{labelformat=empty}
        \inprelimg[width=\textwidth]{Theme.png}
        \captionof{figure}{Themeeinstellungen}
        \captionsetup[figure]{labelformat=default}
    \end{minipage}
\end{minipage}