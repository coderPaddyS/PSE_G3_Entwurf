\subsubsection{Theme}
\paragraph*{Typ}
Enum.
\paragraph*{Beschreibung}
Die Elemente dieses Enums sind Anzeige-Themes.\\
Die konkreten Farbwerte werden gesondert, unter Ressources als ein Compose-Theme gespeichert 
(dies ist eine XML-Datei mit allen Farbwerten). Die konkrete Zuordnung des Enums auf das jeweilige 
Compose-Theme bleibt daher der Implementierung vorbehalten.

\paragraph*{Werte}
\begin{itemize}
    \item \textbf{LIGHT\_MODE}: Hellmodus der App.
    \item \textbf{DARK\_MODE}: Dunkelmodus der App.
\end{itemize}

\subsubsection{\texttt{+ getName() : String}}%$$$M
\paragraph*{Beschreibung}
Gibt den Namen des Themes auf der aktuellen Sprache zurück.
\paragraph*{Parameter}
Keine.
\paragraph*{Rückgabewert}
String,  Namen des Themes auf der aktuellen Sprache.