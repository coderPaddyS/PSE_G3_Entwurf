\subsubsection{Shared Preferences}\label{App_DataManagement_SharedPreferences}
Einige Daten werden über Shared Preferences gespeichert. Es werden nur Strings unterstützt. 
Nachfolgend ist angegeben, welchen Datentyp der gespeicherte String repräsentiert.\\
Folgende Daten werden über Shared Preferences gespeichert:

\begin{itemize}
    \item Einstellungen: 
    \begin{itemize}
        \item \texttt{language : String} Identifier für die ausgewählte Sprache, die in der App verwendet wird.
        \item \texttt{theme : String} Identifier für das ausgewählte Theme, das in der App verwendet wird.
        \item \texttt{storeSearches : boolean} Wert, ob die letzten Suchbegriffe im Suchverlauf gespeichert werden.
    \end{itemize}
    \item Positionsdaten:
    \begin{itemize}
        \item \texttt{lastPosLongitude : double} Längengrad der gespeicherten letzten Position.
        \item \texttt{lastPosLatitude : double} Breitengrad der gespeicherten letzten Position.
    \end{itemize}
    \item Datenbankversionen:
    \begin{itemize}
        \item \texttt{globalAliasVersion : int} Version der lokalen Kopie der globalen Alias-Tabelle.
    \end{itemize}
    \item Anmeldedaten
    \begin{itemize}
        \item \texttt{isSignedIn : boolean} Wert, ob sich der Benutzer angemeldet und noch nicht wieder abgemeldet hat.
        \item \texttt{plattform : String} Anmeldedienst, mit der sich der Benutzer angemeldet hat.
        \item \texttt{accessToken : String} AccessToken.
        \item \texttt{idToken : JWT} Vom Anmeldedienst übermitteltes ID-Token.
        \item \texttt{refreshToken : String} Refresh-Token.
        \item \texttt{userId : String} Vom Server vergebene UserId.
    \end{itemize}
\end{itemize}
