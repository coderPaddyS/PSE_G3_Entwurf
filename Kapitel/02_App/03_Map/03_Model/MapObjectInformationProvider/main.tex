\subsubsection{MapObjectInformationProvider}
\paragraph*{Typ}
Klasse.
\paragraph*{Beschreibung}
Diese Klasse stellt Informationen zu Kartenobjekten bereit.

\subsubsection*{\texttt{+ getBuildingAddress(mapId : int) : String}}%$$$M
\paragraph*{Beschreibung}
Gibt die Adresse eines Gebäudes zurück.
\paragraph*{Parameter}
\begin{itemize}
    \item mapId : int MapId des Gebäudes.
\end{itemize}
\paragraph*{Rückgabewert}
String mit Adresse, oder \texttt{null}, wenn MapId kein Gebäude ist.

\subsubsection*{\texttt{+ getRoomData(mapId : int) : (RoomType, String, int)}}%$$$M
\paragraph*{Kurzbeschreibung}
Gibt Daten zu einem Raum zurück.
\paragraph*{Beschreibung}
Diese Methode gibt zu einem, durch seine MapId gegebenen, Raum folgende Informationen zurück:
\begin{itemize}
    \item Der Typ des Raumes,
    \item die Nummer des Gebäudes, in dem sich der Raum befindet,
    \item das Stockwerk, auf dem sich der Raum befindet.
\end{itemize}
Ist die gegebene MapId ungültig oder kein Raum, wird \texttt{null} zurückgegeben.
\paragraph*{Parameter}
\begin{itemize}
    \item mapId : int MapId des Raums.
\end{itemize}
\paragraph*{Rückgabewert}
(RoomType, String, int, List<String>) Raumdaten oder \texttt{null}.

\subsubsection*{\texttt{+ getPersons(mapId : int) : List<String>}}%$$$M
\paragraph*{Kurzbeschreibung}
Gibt die Namen der Personen zurück, die zu einem Kartenobjekt gehören.
\paragraph*{Beschreibung}
Gibt eine Liste mit den vollen Namen (inklusive Titel) aller dem Kartenobjekt zugeordneten 
Personen zurück. Diese kann leer sein.\\
Ist die gegebene MapId ungültig ist sie ebenfalls leer.
\paragraph*{Parameter}
\begin{itemize}
    \item mapId : int MapId des Kartenobjekts.
\end{itemize}
\paragraph*{Rückgabewert}
List<String> mit den Namen aller zugeordneten Personen.

\subsubsection*{Bilder}
\begin{minipage}{\linewidth}
    \centering
    \begin{minipage}{.49\textwidth}
        \captionsetup[figure]{labelformat=empty}
        \inprelimg[width=\textwidth]{Gebaudeinformation.png}
        \captionof{figure}{Gebäudeinformationen}
        \captionsetup[figure]{labelformat=default}
    \end{minipage}
    \begin{minipage}{.49\textwidth}
        \captionsetup[figure]{labelformat=empty}
        \inprelimg[width=\textwidth]{Karte_mit_Gebaudeinformation_Final.png}
        \captionof{figure}{Etagenansicht mit Gebäudeinformationen}
        \captionsetup[figure]{labelformat=default}
    \end{minipage}
\end{minipage}