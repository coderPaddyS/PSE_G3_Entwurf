\subsubsection{IMapDataProvider}\label{App_Map_Model_IMapDataProvider}
\paragraph*{Typ}
\texttt{Interface}.
\paragraph*{Beschreibung}
Dieses Interface beschreibt die Schnittstelle von Model um Kartendaten anzufordern und zu speichern.
Dieses Interface ist eine Fassade für die Model-Schicht der Karte.

\subsubsection*{\texttt{+ getRoomDisplayData(mapId : int, level : int) : LiveData<List<DisplayData>>}}\label{App_Map_Model_getRoomDisplayData_IMap}%$$$M
\paragraph*{Kurzbeschreibung}
Gibt die Anzeigedaten für Räume einer Etage zurück.
\paragraph*{Beschreibung}
Gibt \hyperref[App_Map_Util_DisplayData]{DisplayData}-Objekte zu allen Räumen einer Etage eines Gebäudes zurück.\\
Die Liste ist leer, wenn es keine hinterlegten Räume gibt.
\paragraph*{Parameter}
\begin{itemize}
    \item \texttt{mapId : int} MapId des Gebäudes.
    \item \texttt{level : int} Stockwerk.
\end{itemize}
\paragraph*{Rückgabewert}
\texttt{LiveData<List<DisplayData>>}, \hyperref[App_Map_Util_DisplayData]{DisplayData} für Räume einer Etage.

\subsubsection*{\texttt{+ getFloorMapTileSource(mapId : int, level : int) : \href{https://osmdroid.github.io/osmdroid/javadocAll/org/osmdroid/tileprovider/tilesource/ITileSource.html}
{ITileSource}}}\label{App_Map_Model_getFloorMapTileSource_IMap}%$$$M
\paragraph*{Beschreibung}
Gibt eine \href{https://osmdroid.github.io/osmdroid/javadocAll/org/osmdroid/tileprovider/tilesource/ITileSource.html}
{ITileSource} mit den Kartendaten für das Stockwerk eines Gebäudes zurück.\\
Existiert die angeforderte Etagenkarte nicht auf dem Gerät, wird \texttt{null} zurückgegeben.
\paragraph*{Parameter}
\begin{itemize}
    \item \texttt{mapId : int} MapId des Gebäudes.
    \item \texttt{level : int} Stockwerk.
\end{itemize}
\paragraph*{Rückgabewert}
\href{https://osmdroid.github.io/osmdroid/javadocAll/org/osmdroid/tileprovider/tilesource/ITileSource.html}
{ITileSource} mit den Kartendaten für die Etage oder \texttt{null}.

\subsubsection*{\texttt{+ getBuildingDisplayData() : LiveData<List<DisplayData>>}}\label{App_Map_Model_getBuildingDisplayData_IMap}%$$$M
\paragraph*{Beschreibung}
Gibt die Anzeigedaten für alle Gebäude zurück.
\paragraph*{Parameter}
Keine.
\paragraph*{Rückgabewert}
\texttt{LiveData<List<DisplayData>>}, \hyperref[App_Map_Util_DisplayData]{DisplayData} für alle Gebäude.

\subsubsection*{\texttt{+ getBuildingTileSource() : \href{https://osmdroid.github.io/osmdroid/javadocAll/org/osmdroid/tileprovider/tilesource/ITileSource.html}
{ITileSource}}}\label{App_Map_Model_getBuildingTileSource_IMap}%$$$M
\paragraph*{Beschreibung}
Gibt eine \href{https://osmdroid.github.io/osmdroid/javadocAll/org/osmdroid/tileprovider/tilesource/ITileSource.html}
{ITileSource} mit den Kartendaten für alle Gebäude zurück.
\paragraph*{Parameter}
\begin{itemize}
    \item \texttt{mapId : int} MapId des Gebäudes.
    \item \texttt{level : int} Stockwerk.
\end{itemize}
\paragraph*{Rückgabewert}
\href{https://osmdroid.github.io/osmdroid/javadocAll/org/osmdroid/tileprovider/tilesource/ITileSource.html}
{ITileSource} mit den Kartendaten für alle Gebäude.

\subsubsection*{\texttt{+ getLastSavedPosition() : IGeoPoint}}\label{App_Map_Model_getLastSavedPosition_IMap}%$$$M
\paragraph*{Kurzbeschreibung}
Gibt die letzte gespeicherte Position zurück.
\paragraph*{Beschreibung}
Diese Methode gibt die letzte gespeicherte Position zurück. Wurde noch keine Position gespeichert,
wird eine Default-Position zurückgegeben. Es wird nie \texttt{null} zurückgegeben.
\paragraph*{Parameter}
Keine.
\paragraph*{Rückgabewert}
\texttt{IGeoPoint} mit letzter gespeicherter Position (oder Default-Wert), ist nie \texttt{null}.

\subsubsection*{\texttt{+ savePosition(pos : IGeoPoint)}}\label{App_Map_Model_savePosition_IMap}%$$$M
\paragraph*{Beschreibung}
Speichert eine Position als letzte Position.
\paragraph*{Parameter}
\begin{itemize}
    \item \texttt{pos : IGeoPoint} zu speichernde Position.
\end{itemize}
\paragraph*{Rückgabewert}
\texttt{Void}.

\subsubsection*{\texttt{+ getBuildingAddress(mapId : int) : String}}\label{App_Map_Model_getBuildingAddress_IMap}%$$$M
\paragraph*{Beschreibung}
Gibt die Adresse eines Gebäudes zurück.
\paragraph*{Parameter}
\begin{itemize}
    \item \texttt{mapId : int} MapId des Gebäudes.
\end{itemize}
\paragraph*{Rückgabewert}
\texttt{String} mit Adresse, oder \texttt{null}, wenn MapId kein Gebäude ist.

\subsubsection*{\texttt{+ getRoomData(mapId : int) : (RoomType, String, int)}}\label{App_Map_Model_getRoomData_IMap}%$$$M
\paragraph*{Kurzbeschreibung}
Gibt Daten zu einem Raum zurück.
\paragraph*{Beschreibung}
Diese Methode gibt zu einem, durch seine MapId gegebenen, Raum folgende Informationen zurück:
\begin{itemize}
    \item Der Typ des Raumes,
    \item die Nummer des Gebäudes, in dem sich der Raum befindet,
    \item das Stockwerk, auf dem sich der Raum befindet.
\end{itemize}
Ist die gegebene MapId ungültig oder kein Raum, wird \texttt{null} zurückgegeben.
\paragraph*{Parameter}
\begin{itemize}
    \item \texttt{mapId : int} MapId des Raums.
\end{itemize}
\paragraph*{Rückgabewert}
\texttt{(RoomType, String, int, List<String>)} Raumdaten oder \texttt{null}.

\subsubsection*{\texttt{+ getPersons(mapId : int) : List<String>}}\label{App_Map_Model_getPersons_IMap}%$$$M
\paragraph*{Kurzbeschreibung}
Gibt die Namen der Personen zurück, die zu einem Kartenobjekt gehören.
\paragraph*{Beschreibung}
Gibt eine Liste mit den vollen Namen (inklusive Titel) aller dem Kartenobjekt zugeordneten 
Personen zurück. Diese kann leer sein.\\
Ist die gegebene MapId ungültig ist sie ebenfalls leer.
\paragraph*{Parameter}
\begin{itemize}
    \item \texttt{mapId : int} MapId des Kartenobjekts.
\end{itemize}
\paragraph*{Rückgabewert}
\texttt{List<String>} mit den Namen aller zugeordneten Personen.
