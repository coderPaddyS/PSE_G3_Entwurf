\subsubsection{LevelModeController}
\paragraph*{Typ}
Implementiert MapLayerDataController, LevelChangeController.
\paragraph*{Beschreibung}
Diese Klasse steuert den Etagenmodus. Sie stellt alle benötigten Daten bereit, um die aktuelle Etage anzuzeigen. 
Sie stellt alle benötigten Aufrufe bereit, um Interaktionen mit der Etagenkarte zu verarbeiten.

\subsubsection*{\texttt{+ <<create>> LevelModeController(dataProvider : IMapDataProvider)}}%$$$M
\paragraph*{Beschreibung}
Erstellt einen neuen LevelModeController.
\paragraph*{Parameter}
\begin{itemize}
    \item \texttt{dataProvider : IMapDataProvider} IMapDataProvider von dem diese Klasse die Kartendaten bezieht.
\end{itemize}

\subsubsection*{\texttt{+ startLevelMode(mapId : int)}}%$$$M
\paragraph*{Kurzbeschreibung}
Startet die Etagenansicht eines Gebäudes.
\paragraph*{Beschreibung}
Diese Methode startet für die Etagenansicht für ein gegebenes Gebäude.\\
Existiert die Etagenansicht für das Gebäude nicht, passiert nichts.\\
Existiert die Etagenansicht für das Gebäude werden alle durch das MapLayerDataController-Interface beobachtbaren 
Kartendaten zur Etagen-Layer aktualisiert. Insbesondere ist isLevelMode() \texttt{wahr}.\\
Ist dieses Gebäude das letzte, von dem die Etagenansicht in dieser Kartenmodus-Session geladen wurde, wird die 
Etage geladen, die zuletzt von diesem Gebäude betrachtet wurde.
Ist dies nicht der Fall oder wurde in dieser Kartenmodus-Session noch keine Etagenansicht geladen, wird das Erdgeschoss geladen.
\paragraph*{Parameter}
\begin{itemize}
    \item \texttt{mapId : int} MapId des Gebäudes.
\end{itemize}
\paragraph*{Rückgabewert}
\texttt{Void}.
