\subsubsection{IMapObjectInfoController}
\paragraph*{Typ}
Interface.
\paragraph*{Beschreibung}
Dieses Interface definiert eine Schnittstelle, die alle Daten bereitstellt, 
die zum Anzeigen der Informationen zu einem Kartenobjekt benötigt werden. 
Außerdem bietet es Methoden an, um einige dieser Daten zu ändern.

\subsubsection*{\texttt{+ getBuildingAddress(mapId : int) : String}}%$$$M
\paragraph*{Beschreibung}
Gibt die Adresse eines Gebäudes zurück.
\paragraph*{Parameter}
\begin{itemize}
    \item mapId : int MapId des Gebäudes.
\end{itemize}
\paragraph*{Rückgabewert}
String mit Adresse, \texttt{null}, wenn MapId ungültig.

\subsubsection*{\texttt{+ getRoomInfo(mapId : int) : (RoomType, String, int, List<String>)}}%$$$M
\paragraph*{Kurzbeschreibung}
Gibt die statischen Informationen zu einem Raum zurück.
\paragraph*{Beschreibung}
Diese Methode gibt zu einem, durch seine MapId gegebenen, Raum folgende Informationen zurück:
\begin{itemize}
    \item Der Typ des Raumes,
    \item die Nummer des Gebäudes, in dem sich der Raum befindet,
    \item das Stockwerk, auf dem sich der Raum befindet,
    \item eine Liste mit den Namen aller Personen, die diesem Raum zugeordnet sind (kann leer sein).
\end{itemize}
Ist die gegebene MapId ungültig, wird \texttt{null} zurückgegeben.
\paragraph*{Parameter}
\begin{itemize}
    \item mapId : int MapId des Raums.
\end{itemize}
\paragraph*{Rückgabewert}
(RoomType, String, int, List<String>), 4-Tupel aus Typ des Raums, Nummer des Gebäudes, Stockwerk, Personen. Kann \texttt{null} sein.

\subsubsection*{\texttt{+ getGlobalAliases(mapId : int) : List<String>}}%$$$M
\paragraph*{Beschreibung}
Gibt eine Liste aller globalen Aliasse eines Kartenobjekts zurück.
\paragraph*{Parameter}
\begin{itemize}
    \item mapId : int MapId des Kartenobjekts.
\end{itemize}
\paragraph*{Rückgabewert}
List<String>, Liste der globalen Aliasse.

\subsubsection*{\texttt{+ getLocalAliases(mapId : int) : LiveData<List<String>>}}%$$$M
\paragraph*{Beschreibung}
Gibt eine Liste aller lokalen Aliasse eines Kartenobjekts zurück.
\paragraph*{Parameter}
\begin{itemize}
    \item mapId : int MapId des Kartenobjekts.
\end{itemize}
\paragraph*{Rückgabewert}
LiveData<List<String>>, Liste der lokalen Aliasse.

\subsubsection*{\texttt{+ deleteLocalAlias(mapId : int, alias : String)}}%$$$M
\paragraph*{Kurzbeschreibung}
Sorgt dafür, dass ein gegebener lokaler Alias gelöscht wird.
\paragraph*{Beschreibung}
Sind die Parameter ungültig, passiert nichts.
\paragraph*{Parameter}
\begin{itemize}
    \item mapId : int MapId des Kartenobjekts.
    \item alias : String Alias der zu löschen ist.
\end{itemize}
\paragraph*{Rückgabewert}
Void.

\subsubsection*{\texttt{+ getAliasSuggestions(mapId : int) : LiveData<List<String>>}}%$$$M
\paragraph*{Kurzbeschreibung}
Gibt eine Liste mit bewertbaren Alias-Vorschlägen zurück.
\paragraph*{Beschreibung}
Gibt eine Liste mit relevanten Alias-Vorschlägen für das Kartenobjekt zurück, 
die vom Benutzer noch bewertet werden dürfen.
\paragraph*{Parameter}
\begin{itemize}
    \item mapId : int MapId des Kartenobjekts.
\end{itemize}
\paragraph*{Rückgabewert}
LiveData<List<String>>, Liste der Alias-Vorschläge.

\subsubsection*{\texttt{+ giveAliasFeedback(mapId : int, suggestion : String, feedback : SuggestionFeedback) : boolean}}%$$$M
\paragraph*{Kurzbeschreibung}
Diese Methode sorgt dafür, dass Feedback zu einem Alias-Vorschlag gesendet wird.
\paragraph*{Beschreibung}
Die Methode prüft, ob der Benutzer das Feedback senden kann und darf.
Ist der Benutzer nicht angemeldet, wird in den Anmeldemodus gewechselt. Das Senden des Feedbacks entfällt dann.\\
Muss der Benutzer sich für diese Aktion erst erneut anmelden, wird in den Anmeldemodus zur erneuten Anmeldung gewechselt. Das Senden des Feedbacks entfällt dann.\\
Das Feedback wird an das Model gegeben, um es an den Server zu senden.
\paragraph*{Parameter}
\begin{itemize}
    \item mapId : int MapId des Kartenobjekts.
    \item suggestion : String Vorschlag. 
    \item feedback : SuggestionFeedback Feedback.
\end{itemize}
\paragraph*{Rückgabewert}
boolean, ob das Senden erfolgreich war.