\subsubsection{MapLayerDataController}\label{App_Map_ViewModel_MapLayerDataController}
\paragraph*{Typ} 
\texttt{Interface}
\paragraph*{Beschreibung}
Dieses Interface definiert eine Schnittstelle, die Daten für eine Kartenebene bereitstellt.
Implementierende Klassen können von einem Beobachter beobachtet werden. Dazu werden die 
Daten für die Karte als LiveData-Objekte herausgegeben.\\
Außerdem wird die Änderung dieser Daten spezifiziert, wenn ein Objekt ausgewählt wird, dessen
Daten im \hyperref[App_Map_ViewModel_MapLayerDataController]{MapLayerDataController} verwaltet werden.

\subsubsection*{\texttt{+ selectMapObject(mapId : int)}}\label{App_Map_ViewModel_selectMapObject}%$$$M
\paragraph*{Kurzbeschreibung}
Diese Methode fokussiert das richtige Element, nachdem ein Kartenobjekt ausgewählt wurde.
\paragraph*{Beschreibung}
Diese Methode wird aufgerufen, wenn ein Kartenobjekt vom Benutzer ausgewählt wird, dessen 
Daten von diesem \hyperref[App_Map_ViewModel_MapLayerDataController]{MapLayerDataController} verwaltet werden.\\
Diese Methode kümmert sich um die Fokussierung des richtigen Kartenobjekts. Eventuell muss 
aus oder in die Etagenansicht gewechselt werden.
\paragraph*{Parameter}
\begin{itemize}
    \item \texttt{mapId : int} MapId des ausgewählten Kartenobjekts.
\end{itemize}
\paragraph*{Rückgabewert}
\texttt{Void}.

\subsubsection*{\texttt{+ getTileSource() : LiveData<\href{https://osmdroid.github.io/osmdroid/javadocAll/org/osmdroid/tileprovider/tilesource/ITileSource.html}{ITileSource}>}}\label{App_Map_ViewModel_getTileSource}%$$$M
\paragraph*{Kurzbeschreibung}
Gibt die \href{https://osmdroid.github.io/osmdroid/javadocAll/org/osmdroid/tileprovider/tilesource/ITileSource.html}
{ITileSource} mit den Daten für eine Kartenebene zurück.
\paragraph*{Beschreibung}
Gibt \texttt{null} zurück, wenn gar nichts angezeigt werden soll.
\paragraph*{Parameter}
Keine.
\paragraph*{Rückgabewert}
\texttt{LiveData<\href{https://osmdroid.github.io/osmdroid/javadocAll/org/osmdroid/tileprovider/tilesource/ITileSource.html}
{ITileSource}>} mit den Kartendaten, darf \texttt{null} sein.

\subsubsection*{\texttt{+ getMapObjectData() : LiveData<List<DisplayData>>}}\label{App_Map_ViewModel_getMapObjectData}%$$$M
\paragraph*{Beschreibung}
Gibt die Anzeige-Daten der Kartenobjekte zurück.
\paragraph*{Parameter}
Keine.
\paragraph*{Rückgabewert}
\texttt{LiveData<List<DisplayData>>} mit den Daten der Kartenobjekte.

\subsubsection*{\texttt{+ getFocusedMapId() : LiveData<int>}}\label{App_Map_ViewModel_getFocusedMapId}%$$$M
\paragraph*{Beschreibung}
Gibt die MapId des fokussierten Kartenobjekts zurück.
\paragraph*{Parameter}
Keine.
\paragraph*{Rückgabewert}
\texttt{LiveData<int>} MapId des fokussierten Kartenobjekts.

\subsubsection*{\texttt{+ isLevelMode() : LiveData<boolean>}}\label{App_Map_ViewModel_isLevelMode}%$$$M
\paragraph*{Kurzbeschreibung}
Gibt zurück, ob das fokussierte Kartenobjekt auf einer Etage ist.
\paragraph*{Beschreibung}
Ist wahr, wenn das fokussierte Kartenobjekt ein Kartenobjekt ist, das im Etagenmodus 
angezeigt wird. Sonst wird false zurückgegeben.
\paragraph*{Parameter}
Keine.
\paragraph*{Rückgabewert}
\texttt{LiveData<boolean>} ob das fokussierte Kartenobjekt auf einer Etage ist.