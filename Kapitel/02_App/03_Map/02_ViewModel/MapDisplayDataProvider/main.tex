\subsubsection{MapDisplayDataProvider}
\paragraph*{Typ}
Interface.
\paragraph*{Beschreibung}
Dieses Interface ist eine Schnittstelle für Lade- und Speicheroperationen zur angezeigten Karte.\\
Klassen dieses Interfaces können von einem Beobachter beobachtet werden. Dazu werden 
LiveData-Klassen benutzt.

\subsubsection*{\texttt{+ getTileSource() : LiveData<ITileSource>}}%$$$M
\paragraph*{Beschreibung}
Gibt die ITileSource mit den Kartendaten zurück.
\paragraph*{Parameter}
Keine.
\paragraph*{Rückgabewert}
LiveData<TileSource> mit den Kartendaten.

\subsubsection*{\texttt{+ saveLastPos(pos : IGeoPoint)}}%$$$M
\paragraph*{Beschreibung}
Speichert eine gegebene Position als letzte Position.
\paragraph*{Parameter}
\begin{itemize}
    \item pos : IGeoPoint zu speichernde Position.
\end{itemize}
\paragraph*{Rückgabewert}
Void.

\subsubsection*{\texttt{+ getLastPos() : IGeoPoint}}%$$$M
\paragraph*{Kurzbeschreibung}
Gibt die letzte Position zurück.
\paragraph*{Beschreibung}
Diese Methode gibt die als letzte Position gespeicherte Position als IGeoPoint zurück.
Wurde noch keine Position gespeichert, wird eine Defaut-Position zurückgegeben.
Dies wird benötigt, damit die Karte beim Neuladen die gleiche Stelle zeigen kann.
\paragraph*{Parameter}
Keine.
\paragraph*{Rückgabewert}
IGeoPoint, letzte Position.

\subsubsection*{\texttt{+ getFocusedPos() : LiveData<IGeoPoint>}}%$$$M
\paragraph*{Kurzbeschreibung}
Gibt die Position zurück, die mittig auf der Karte angezeigt werden muss.
\paragraph*{Beschreibung}
Wird ein Objekt auf der Karte fokussiert, und die Karte muss diese Position mittig auf dem 
Bildschirm anzeigen (z.B. um ein Suchergebnis anzuzeigen), so wird dieser IGeoPoint aktualisiert.\\
Darf null sein, wenn keine bestimmte Position fokussiert werden muss.
\paragraph*{Parameter}
Keine.
\paragraph*{Rückgabewert}
LiveData<IGeoPoint>, Position, die mittig angezeigt werden soll.

\subsubsection*{\texttt{+ getBuildingController() : MapLayerDataController}}%$$$M
\paragraph*{Beschreibung}
Gibt den MapLayerDataController für die Gebäude-Ebene der Karte zurück.
\paragraph*{Parameter}
Keine.
\paragraph*{Rückgabewert}
MapLayerDataController für die Gebäude-Ebene der Karte.

\subsubsection*{\texttt{+ getLevelController() : MapLayerDataController}}%$$$M
\paragraph*{Beschreibung}
Gibt den MapLayerDataController für die Etagen-Ebene auf der Karte zurück.
\paragraph*{Parameter}
Keine.
\paragraph*{Rückgabewert}
MapLayerDataController für die Etagen-Ebene auf der Karte.

\subsubsection*{\texttt{+ getLevelChangeController() : LevelChangeController}}%$$$M
\paragraph*{Beschreibung}
Gibt den LevelChangeControllers für den Etagenwechsel zurück.
\paragraph*{Parameter}
Keine.
\paragraph*{Rückgabewert}
LevelChangeController für den Etagenwechsel.

\subsubsection*{\texttt{+ getMapObjectInfoController() : IMapObjectInfoController}}%$$$M
\paragraph*{Beschreibung}
Gibt den IMapObjectInfoController zurück.
\paragraph*{Parameter}
Keine.
\paragraph*{Rückgabewert}
IMapObjectInfoController.

\subsubsection*{\texttt{+ getRoomError() : LiveData<(String, String, String)>}}%$$$M
\paragraph*{Kurzbeschreibung}
Gibt die Error-Daten zurück, wenn ein Raum nicht in der App existiert.
\paragraph*{Beschreibung}
Wird das Anzeigen eines Raums angefordert, der in der App nicht existiert (das Gebäude aber schon), 
muss eine Fehlermeldung erscheinen. Die Daten für diese Fehlermeldung liefert diese Methode.\\
Es werden Gebäude Nummer, angeforderter Raum und vermutetes Stockwerk, in dem sich der Raum befindet, 
falls er existiert, in dieser Reihenfolge zurückgegeben.\\
Gibt es keine Fehlerdaten, da es keinen Fehler gibt, wird LiveData<null> zurückgegeben.
\paragraph*{Parameter}
Keine.
\paragraph*{Rückgabewert}
LiveData<(String, String, String)> aus Gebäudenummer, Raumnummer, Stockwerk.
