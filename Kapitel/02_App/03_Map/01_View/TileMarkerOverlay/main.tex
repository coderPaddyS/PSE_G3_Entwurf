\subsubsection{TileMarkerOverlay}\label{App_Map_View_TileMarkerOverlay}
\paragraph*{Typ} 
Abstrakte Klasse, erbt von Overlay.
\paragraph*{Entwurfsmuster}
\subparagraph*{Dekorierer}
Diese Klasse ist ein leicht abgewandelter Dekorierer von Overlay.
Die Abwandlung besteht darin, dass das Objekt, an das delegiert wird nicht ein
beliebiges Overlay, sondern ein FolderOverlay ist. Außerdem werden die neuen 
Methoden direkt in dieser Klasse hinzugefügt.\\
\subparagraph*{Beobachter}
Diese Klasse ist ein Beobachter. Das beobachtete Subjekt ist hier ein \hyperref[App_Map_ViewModel_MapLayerDataController]{MapLayerDataController}.
Die Beobachtung geschieht über die LiveData Mechanismen.
\paragraph*{Beschreibung}
Dieses Overlay besitzt ein TileOverlay (also eine Kartenebene) und eine 
Liste von \hyperref[App_Map_View_MapObjectMarker]{MapObjectMarkern}. Diese werden über dem TileOverlay angezeigt.
Für die Verwaltung dieser Overlays wird das FolderOverlay eingesetzt.\\
Diese Klasse verwaltet die \hyperref[App_Map_View_MapObjectMarker]{MapObjectMarkern}, die Aktion beim Klicken auf diese Marker, 
und das Fokussieren eines dieser Marker.

\subsubsection*{\texttt{+ <<create>> \hyperref[App_Map_View_TileMarkerOverlay]{TileMarkerOverlay}(controller : \hyperref[App_Map_ViewModel_MapLayerDataController]{MapLayerDataController}, mapView : MapView)}}%$$$M
\paragraph*{Kurzbeschreibung}
Erstellt eine \hyperref[App_Map_View_TileMarkerOverlay]{TileMarkerOverlay}-Instanz.
\paragraph*{Beschreibung}
Beginnt die von controller bereitgestellten Daten zu beobachten. Erzeugt das TileOverlay.
Erzeugt die \hyperref[App_Map_View_MapObjectMarker]{MapObjectMarkern} über die Fabrikmethode createMarker.
\paragraph*{Parameter}
\begin{itemize}
    \item \texttt{controller : \hyperref[App_Map_ViewModel_MapLayerDataController]{MapLayerDataController}} \hyperref[App_Map_ViewModel_MapLayerDataController]{MapLayerDataController} der die Daten bereitstellt.
    \item \texttt{mapView : MapView} MapView auf der sich das Overlay befindet.
\end{itemize}

\subsubsection*{Delegierte Methoden}
Für das Dekorierer-Entwurfsmuster werden folgende Methoden von Overlay ohne Veränderung 
an die FolderOverlay-Instanz delegiert:
\begin{itemize}
    \item \texttt{draw(pCanvas : Canvas, pProjection : Projection)}
    \item \texttt{draw(pCanvas : Canvas, pMapView : MapView, pShadow : boolean)}
    \item \texttt{onSingleTapUp(e : MotionEvent, mapView : MapView) : boolean}
    \item \texttt{onSingleTapConfirmed(e : MotionEvent, mapView : MapView) : boolean}
    \item \texttt{onLongPress(e : MotionEvent, mapView : MapView) : boolean}
    \item \texttt{onTouchEvent(e : MotionEvent, mapView : MapView) : boolean}
    \item \texttt{onDetach(mapView : MapView)}
\end{itemize}

\subsubsection*{\texttt{+ onMarkerTap(mapId : int)}}%$$$M
\paragraph*{Beschreibung}
Teilt dem \hyperref[App_Map_ViewModel_MapLayerDataController]{MapLayerDataController} mit, dass ein bestimmtes Kartenobjekt ausgewählt wurde.
\paragraph*{Parameter}
\begin{itemize}
    \item \texttt{mapId : int} MapId des Kartenobjekts, auf das ausgewählt wurde.
\end{itemize}
\paragraph*{Rückgabewert}
\texttt{Void}.

\subsubsection*{\texttt{\# createMarker(data : \hyperref[App_Map_Util_DisplayData]{DisplayData}, mapView : MapView) : \hyperref[App_Map_View_MapObjectMarker]{MapObjectMarkern} $\lbrace$abstract$\rbrace$}}%$$$M
\paragraph*{Beschreibung}
Dies ist eine Fabrikmethode zur Erzeugung eines \hyperref[App_Map_View_MapObjectMarker]{MapObjectMarkern}.
\paragraph*{Parameter}
\begin{itemize}
    \item \texttt{data : DisplayData} \hyperref[App_Map_Util_DisplayData]{DisplayData} mit den Anzeige-Daten für den Marker.
    \item \texttt{mapView : MapView} MapView auf der der Marker angezeigt wird.
\end{itemize}
\paragraph*{Rückgabewert}
\hyperref[App_Map_View_MapObjectMarker]{MapObjectMarkern}.


\subsubsection*{Weitere private Funktionalität des Beobachters}
Ändern sich die beobachteten Daten aktualisieren sich die Overlays wie folgt:
\begin{itemize}
    \item TileSource: Das TileOverlay wird mit den neuen Kartendaten aktualisiert.
    Ist TileSource \texttt{null}, wird gar nichts angezeigt (also auch keine Marker).
    \item MapObjectData: Alle Marker werden verworfen und mit den neuen Daten neu erzeugt (über createMarker).
    \item FocusedMapId und FloorFocus: Der aktuell fokussierte Marker wird zurückgesetzt.
    Dem Marker mit der MapId FocusedMapId wird über seine focusMarkerObject-Methode FloorFocus übergeben.
\end{itemize}

%-------------------------------------------------------------
\subsubsection{BuildingOverlay}
\paragraph*{Typ}
Erbt von \hyperref[App_Map_View_TileMarkerOverlay]{TileMarkerOverlay}.
\paragraph*{Beschreibung}
Diese Klasse ist ein \hyperref[App_Map_View_TileMarkerOverlay]{TileMarkerOverlay} für die Gebäudeebene auf der Karte.

\subsubsection*{\texttt{\# createMarker(data : \hyperref[App_Map_Util_DisplayData]{DisplayData}) : \hyperref[App_Map_View_MapObjectMarker]{MapObjectMarkern}}}%$$$M
\paragraph*{Kurzbeschreibung}
Erzeugt einen \hyperref[App_Map_View_BuildingMarker]{BuildingMarker} mit den gegeben Daten.
\paragraph*{Beschreibung}
Die Fabrikmethode zur Erzeugung eines Markers wird hier mit einem \hyperref[App_Map_View_BuildingMarker]{BuildingMarker} implementiert.
\paragraph*{Parameter}
\begin{itemize}
    \item \texttt{data : \hyperref[App_Map_Util_DisplayData]{DisplayData}} \hyperref[App_Map_Util_DisplayData]{DisplayData} mit den Anzeige-Daten für den Marker.
    \item \texttt{mapView : MapView} MapView auf der der Marker angezeigt wird.
\end{itemize}
\paragraph*{Rückgabewert}
\hyperref[App_Map_View_MapObjectMarker]{MapObjectMarkern}, erzeugter \hyperref[App_Map_View_MapObjectMarker]{MapObjectMarkern}.

\subsubsection*{\texttt{Weitere Implementierungen}}%$$$M
Benutzt den von \hyperref[App_Map_View_TileMarkerOverlay]{TileMarkerOverlay} geerbten Konstruktor.

%-------------------------------------------------------------
\subsubsection{FloorOverlay}
\paragraph*{Typ}
Erbt von \hyperref[App_Map_View_TileMarkerOverlay]{TileMarkerOverlay}.
\paragraph*{Beschreibung}
Diese Klasse ist ein \hyperref[App_Map_View_TileMarkerOverlay]{TileMarkerOverlay} für die Etagenebene auf der Karte.

\subsubsection*{\texttt{\# createMarker(data : \hyperref[App_Map_Util_DisplayData]{DisplayData}) : \hyperref[App_Map_View_MapObjectMarker]{MapObjectMarkern}}}%$$$M
\paragraph*{Kurzbeschreibung}
Erzeugt einen \hyperref[App_Map_View_RoomMarker]{RoomMarker} mit den gegeben Daten.
\paragraph*{Beschreibung}
Die Fabrikmethode zur Erzeugung eines Markers wird hier mit einem \hyperref[App_Map_View_RoomMarker]{RoomMarker} implementiert.
\paragraph*{Parameter}
\begin{itemize}
    \item \texttt{data : \hyperref[App_Map_Util_DisplayData]{DisplayData}} \hyperref[App_Map_Util_DisplayData]{DisplayData} mit den Anzeige-Daten für den Marker.
    \item \texttt{mapView : MapView} MapView auf der der Marker angezeigt wird.
\end{itemize}
\paragraph*{Rückgabewert}
\hyperref[App_Map_View_MapObjectMarker]{MapObjectMarkern}, erzeugter \hyperref[App_Map_View_MapObjectMarker]{MapObjectMarkern}.

\subsubsection*{Weitere Implementierungen}%$$$M
Benutzt den von \hyperref[App_Map_View_TileMarkerOverlay]{TileMarkerOverlay} geerbten Konstruktor.