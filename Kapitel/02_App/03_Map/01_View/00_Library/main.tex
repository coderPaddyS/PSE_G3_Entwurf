\subsubsection{Library Osmdroid}\label{App_Map_View_Library}
Die quelloffene Library \href{https://osmdroid.github.io/osmdroid/}{Osmdroid} stellt Werkzeuge und View-Klassen zur Verfügung, die das Anzeigen, 
Interagieren und Manipulieren von Karten erleichtern.
Mehr über \href{https://osmdroid.github.io/osmdroid/}{Osmdroid} ist hier zu finden:
\href{https://github.com/osmdroid/osmdroid}{Quellcode auf Github}, 
\href{https://osmdroid.github.io/osmdroid/javadoc.html}{Javadoc Dokumentation}
\paragraph*{Wichtige Klassen}
\begin{itemize}
    \item \texttt{\href{https://osmdroid.github.io/osmdroid/javadocAll/org/osmdroid/views/MapView.html}
    {MapView}}: Anzeige der Karte im \href{https://developer.android.com/reference/android/view/View}{Android View System}.
    \item \texttt{\href{https://osmdroid.github.io/osmdroid/javadocAll/org/osmdroid/views/overlay/Overlay.html}
    {Overlay}}: Eine weitere Ebene auf beziehungsweise über der Karte anzeigen.
    \item \texttt{\href{https://osmdroid.github.io/osmdroid/javadocAll/org/osmdroid/views/overlay/Marker.html}
    {Marker}}: Eine Markierung auf der Karte.
    \item \texttt{\href{https://osmdroid.github.io/osmdroid/javadocAll/org/osmdroid/api/IGeoPoint.html}
    {IGeoPoint}}: Interface für einen zweidimensionalen Punkt mit Längen- und Breitengrad.
\end{itemize}
