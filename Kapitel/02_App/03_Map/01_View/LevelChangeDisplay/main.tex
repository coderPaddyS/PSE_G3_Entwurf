\subsubsection{LevelChangeDisplay}
\paragraph*{Typ}
Composable.
\paragraph*{Beschreibung}
Diese Klasse stellt über Compose eine Schaltfläche zum Wechseln der angezeigten Etage bereit.\\
Die Schaltfläche besteht aus:
\begin{itemize}
    \item Einem Button zum Wechseln der Etage nach oben.
    \item Einer Textanzeige, die die Etagennummer der aktuell angezeigten Etage anzeigt 
    (z.B. \dq 1. OG\dq, \dq EG\dq, \dq 1. UG\dq. usw.)
    \item Einem Button zum Wechseln der Etage nach unten.
\end{itemize}
Die Klasse ist ein Beobachter eines LevelChangeControllers. Dadurch erhält das 
LevelChangeDisplay die Informationen zur aktuellen Etage. Wird einer der Buttons zum 
Wechseln der Etage betätigt, meldet diese Klasse dies dem LevelChangeController, 
die sich um den Etagenwechsel kümmert.

\subsubsection*{\texttt{+ onClickUp()}}%$$$M
\paragraph*{Kurzbeschreibung}
Veranlasst den Etagenwechsel nach oben.
\paragraph*{Beschreibung}
Diese Methode wird aufgerufen, wenn der Button nach oben geklickt wurde. \\
Wenn bereits die oberste Etage angezeigt wird passiert nichts. 
Wird nicht die oberste Etage angezeigt, wird der Wechsel nach oben veranlasst.
Dazu wird dies dem LevelChangeController mitgeteilt.
\paragraph*{Parameter}
Keine.
\paragraph*{Rückgabewert}
Void.

\subsubsection*{\texttt{+ onClickDown()}}%$$$M
\paragraph*{Kurzbeschreibung}
Veranlasst den Etagenwechsel nach unten.
\paragraph*{Beschreibung}
Diese Methode wird aufgerufen, wenn der Button nach unten geklickt wurde. \\
Wenn bereits die unterste Etage angezeigt wird passiert nichts. 
Wird nicht die unterste Etage angezeigt, wird der Wechsel nach unten veranlasst.
Dazu wird dies dem LevelChangeController mitgeteilt.
\paragraph*{Parameter}
Keine.
\paragraph*{Rückgabewert}
Void.

\subsubsection*{Weitere private Funktionalität}%$$$M
Beobachtete Daten von LevelChangeController und Funktionalität:
\begin{itemize}
    \item isHighestLevel: Wird die oberste Etage angezeigt, wird der Button nach oben ausgegraut 
    und hat keine Funktion. Wird nicht die oberste Etage angezeigt, ist er nicht ausgegraut und funktioniert.
    \item isLowestLevel: Wird die unterste Etage angezeigt, wird der Button nach unten ausgegraut 
    und hat keine Funktion. Wird nicht die unterste Etage angezeigt, ist er nicht ausgegraut und funktioniert.
    \item getLevel: Ändert sich die Etage, wird die Anzeige der aktuellen Etage aktualisiert.
    \item isLevelMode: Befindet sich die Karte nicht im Etagenmodus, wird die ganze Schaltfläche ausgeblendet.
    Befindet sich die Karte im Etagenmodus, wird die Schaltfläche angezeigt.
\end{itemize}
