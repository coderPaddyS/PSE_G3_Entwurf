\subsubsection{MapObjectMarker}
\paragraph*{Typ}
Abstrakte Klasse, erbt von Marker.
\paragraph*{Beschreibung}
Diese Klasse dient als Basisklasse für alle Kartenobjekte, die angezeigt werden sollen.
Das Verhalten beim Antippen des Markers ist in dieser Klasse definiert.
Diese Klasse benutzt eine Strategie. Das Verhalten, wenn das Kartenobjekt fokussiert wird,
wird in die Subklassen ausgelagert, da dies je nach Kartenobjekttyp variiert.

\subsubsection*{\texttt{+ <<create>> MapObjectMarker(mapView : MapView, data : DisplayData, tileMarkerOverlay : TileMarkerOverlay)}}%$$$M
\paragraph*{Beschreibung}
Erstellt eine neue MapObjectMarker-Instanz.
\paragraph*{Parameter}
\begin{itemize}
    \item \texttt{mapView : MapView} MapView auf der der Marker angezeigt wird.
    \item \texttt{data : DisplayData} Alle Daten, die zur Erstellung des Markers benötigt werden.
    \item \texttt{tileMarkerOverlay : TileMarkerOverlay} TileMarkerOverlay-Instanz, der ein Klick 
    auf diesen Marker mitgeteilt wird. Darf \texttt{null} sein.
\end{itemize}

\subsubsection*{\texttt{+ getMapId() : int}}%$$$M
\paragraph*{Beschreibung}
Gibt MapId des KartenObjekts des Markers zurück.
\paragraph*{Parameter}
Keine.
\paragraph*{Rückgabewert}
\texttt{int}, MapId des Kartenobjekt des Markers.

\subsubsection*{\texttt{+ onSingleTapConfirmed(e : MotionEvent, mapView : MapView) : boolean}}%$$$M
\paragraph*{Beschreibung}
Das Anklicken dieses Markers wird der TileMarkerOverlay-Instanz mitgeteilt.
\paragraph*{Parameter}
Für die Verarbeitung werden beide Parameter nicht benutzt. Diese sind durch Overlay vorgegeben.
\begin{itemize}
    \item \texttt{e : MotionEvent} MotionEvent
    \item \texttt{mapView : MapView} MapView auf der der Marker ist.
\end{itemize}
\paragraph*{Rückgabewert}
\texttt{boolean}, gibt true zurück, da das MotionEvent verarbeitet wurde.

\subsubsection*{\texttt{+ focusMarkerObject(floorMapExists : boolean) $\lbrace$abstract$\rbrace$}}%$$$M
\paragraph*{Beschreibung}
Beschreibt das (graphische) Verhalten des Markers, wenn das zugehörige Objekt fokussiert wird.
\paragraph*{Parameter}
\begin{itemize}
    \item \texttt{floorMapExists : boolean} Genau dann wahr, wenn es zum (evtl. indirekt) ausgewählten Gebäude eine Etagenkarte gibt.
\end{itemize}
\paragraph*{Rückgabewert}
\texttt{Void}.

\subsubsection*{\texttt{+ resetFocus() $\lbrace$abstract$\rbrace$}}%$$$M
\paragraph*{Beschreibung}
Setzt das Aussehen des Markers zurück, wenn er nicht mehr fokussiert wird.
\paragraph*{Parameter}
Keine.
\paragraph*{Rückgabewert}
\texttt{Void}.

%-------------------------------------------------------------
\subsubsection{BuildingMarker}
\paragraph*{Typ} 
Erbt von MapObjectMarker.
\paragraph*{Beschreibung}
MapObjectMarker für Gebäude.

\subsubsection*{\texttt{+ focusMarkerObject(floorMapExists : boolean)}}%$$$M
\paragraph*{Kurzbeschreibung}
Beschreibt das (graphische) Verhalten des Markers, wenn das zugehörige Gebäude fokussiert wird.
\paragraph*{Beschreibung}
Gibt es für das Gebäude eine Etagenkarte, wird der Marker nicht weiter angezeigt.
Gibt es keine Etagenkarte, wird der Marker als großer Standort-Pfeil angezeigt.
\paragraph*{Parameter}
\begin{itemize}
    \item \texttt{floorMapExists : boolean} Genau dann wahr, wenn es zum (evtl. indirekt) ausgewählten Gebäude eine Etagenkarte gibt.
\end{itemize}
\paragraph*{Rückgabewert}
\texttt{Void}.

\subsubsection*{Weitere Implementierungen}
Diese Klasse implementiert ebenfalls die abstrakte Methode resetFocus von MapObjectMarker 
und benutzt den gleichen (geerbten) Konstruktor.

%-------------------------------------------------------------
\subsubsection{RoomMarker}
\paragraph*{Typ} 
Erbt von MapObjectMarker.
\paragraph*{Beschreibung}
MapObjectMarker für Räume.

\subsubsection*{\texttt{+ focusMarkerObject(floorMapExists : boolean)}}%$$$M
\paragraph*{Kurzbeschreibung}
Beschreibt das (graphische) Verhalten des Markers, wenn der zugehörige Raum fokussiert wird.
\paragraph*{Beschreibung}
Der Marker als großer Standort-Pfeil angezeigt.
\paragraph*{Parameter}
\begin{itemize}
    \item \texttt{floorMapExists : boolean} Genau dann wahr, wenn es zum (evtl. indirekt) ausgewählten Gebäude eine Etagenkarte gibt.
\end{itemize}
\paragraph*{Rückgabewert}
\texttt{Void}.

\subsection*{Weitere Implementierungen}
Diese Klasse implementiert ebenfalls die abstrakte Methode resetFocus von MapObjectMarker 
und benutzt den gleichen (geerbten) Konstruktor.
