\subsubsection{DocumentScreen}
\paragraph*{Typ}
Composable.
\paragraph*{Beschreibung}
Klasse zum Anzeigen eines Textdokuments im Markdown-Format.\\
Diese Klasse nutzt die Funktionalität einer Markdown-Library.\\
Das anzuzeigende Dokument wird direkt aus den Ressourcen geladen.
Welches Dokument geladen werden soll, wird wie in Navigation beschrieben, beim Aufruf mitgegeben. 
Dabei wird die Id des DokumentTypes verwendet.

\subsubsection*{Beschreibung der Benutzeroberfläche}
Das Dokument wird im Vollbild angezeigt.
Die Überschrift wird am oberen Bildschirmrand angezeigt.
Links neben der Überschrift am oberen Bildschirmrand ist ein \dq Zurück-Knopf \dq{}.
Der Text wird unter der Unterschrift und dem \dq Zurück-Knopf \dq{} angezeigt und man kann scrollen.

\subsubsection*{\texttt{+ onClickBack()}}%$$$M
\paragraph*{Beschreibung}
Veranlasst bei Betätigung des \dq Zurück-Knopfs \dq{} das Schließen des Textanzeigefensters.
\paragraph*{Parameter}
Keine.
\paragraph*{Rückgabewert}
Void.

\subsubsection*{\texttt{+ onClickReturn()}}%$$$M
\paragraph*{Beschreibung}
Veranlasst bei Betätigung der Zurückgehen-Funktion des Endgeräts das Schließen des Textanzeigefensters.
\paragraph*{Parameter}
Keine.
\paragraph*{Rückgabewert}
Void.

\subsubsection*{Bilder}
\begin{minipage}{\linewidth}
    \centering
    \begin{minipage}{.49\textwidth}
        \captionsetup[figure]{labelformat=empty}
        \inprelimg[width=\textwidth]{Dokument.png}
        \captionof{figure}{Dokumentenansicht}
        \captionsetup[figure]{labelformat=default}
    \end{minipage}
\end{minipage}