\subsection{Wechsel zwischen Ansichten}\label{App_Navigation_Wechsel}
Der Wechsel zwischen den Ansichten wird mithilfe der von \href{https://developer.android.com/jetpack/compose/navigation?hl=fr}
{Jetpack Compose bereitgestellten Navigation mithilfe eines NavHosts und eines NavControllers} 
umgesetzt.

\subsubsection*{Ziele}\label{App_Navigation_Ziele}
Der Wechsel in eine bestimmte Ansicht kann durch folgende Ziele durchgeführt werden:
\begin{itemize}
    \item \Gls{Kartenansicht}: 
    \begin{itemize}
        \item \hyperref[App_Map]{map}: Wechsel in die \Gls{Kartenansicht}, ohne ein bestimmtes Kartenobjekt zu fokussieren.
        \item \hyperref[App_Map]{map}/\texttt{\{mapId\}}: Wechsel in die \Gls{Kartenansicht}, das durch seine MapId gegebene Kartenobjekt wird fokussiert.
        \item \hyperref[App_Map]{map}/\texttt{\{mapId\}/\{number\}}: Wechsel in die \Gls{Kartenansicht}, das durch seine MapId gegebene Gebäude wird fokussiert. 
        Über den durch seine Raumnummer gegebene, in der App nicht existierende Raum wird eine Vermutung zum Stockwerk angezeigt.
    \end{itemize}
    \item \texttt{Alias-Hinzufügen-Ansicht}:
    \begin{itemize}
        \item \texttt{addalias/\{mapId\}}: Wechsel in die \texttt{Alias-Hinzufügen-Ansicht}. 
        Der Alias soll für ein durch seine MapId gegebenes Kartenobjekt hinzugefügt werden.
    \end{itemize}
    \item \hyperref[App_Search]{Suche}: 
    \begin{itemize}
        \item \texttt{search}: Wechsel in die \hyperref[App_Search]{Suche}
    \end{itemize}
    \item \hyperref[App_SignUp]{Anmeldung}: \begin{itemize}
        \item \texttt{signin/new}: Wechsel in die \hyperref[App_SignUp]{Anmeldung}.
        \item \texttt{signin/funktion}: Wechsel in die \hyperref[App_SignUp]{Anmeldung} mit dem Hinweis, dass die gewünschte Funktion nur angemeldeten Benutzern zur Verfügung steht.
        \item \texttt{signin/again}: Wechsel in die \hyperref[App_SignUp]{Anmeldung} mit dem Hinweis, dass die Anmeldung des Benutzers abgelaufen ist und er sich erneut anmelden muss.
    \end{itemize}
    \item \hyperref[App_Settings]{Einstellungen}: 
    \begin{itemize}
        \item \texttt{settings}: Wechsel in die \hyperref[App_Settings]{Einstellungen}
    \end{itemize}
    \item \hyperref[App_DocumentViewer]{Dokumentenanzeige}: 
    \begin{itemize}
        \item \texttt{documentview/\{documentId\}}: Wechsel in die \hyperref[App_DocumentViewer]{Dokumentenanzeige}. 
        Es wird das durch seine ID gegebene Dokument angezeigt. Die IDs sind dem \hyperref[App_DocumentViewer_DocumentType]{Enum DocumentType} zu entnehmen.
    \end{itemize}
\end{itemize}
Zusätzlich erlaubt die Navigation das zurückgehen zur vorherigen Ansicht.


