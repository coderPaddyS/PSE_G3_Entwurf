\section{MapScreen}
\paragraph*{Typ}
Composable.
\paragraph*{Beschreibung}
Die Aufgabe dieses Composables ist das Anzeigen der Karte und der Schaltflächen zur Kartensteuerung.\\
Dazu ist folgendes nötig:
\begin{itemize}
    \item Anzeigen der Karte selbst. Diese besteht aus
        \begin{itemize}
            \item der \href{https://osmdroid.github.io/osmdroid/javadocAll/org/osmdroid/views/MapView.html}
            {MapView} selbst, die u.a. die Straßenkarte anzeigt,
            \item einem TileMarkerOverlay, das die Gebäude anzeigt,
            \item einem TileMarkerOverlay, das wenn die App im Etagenmodus ist, die Etagenkarte anzeigt.
        \end{itemize}
        Die Karte befindet sich hinter allen weiteren Elementen.
    \item Anzeigen des LevelChangeDisplays. Damit kann der Benutzer, wenn die App im Etagenmodus 
    ist, die Etage wechseln.
    \item Anzeigen des Buttons zur Ortung.
    \item Anzeigen eines Kompasses, der die Himmelsrichtungen auf der Karte zeigt.
\end{itemize}

\paragraph*{Parameter}
\begin{itemize}
    \item dataProvider : MapDisplayDataProvider Klasse, die die Hintergrundkarte und Kartenposition bereitstellt.
    \item buildingController : MapLayerDataController MapLayerDataController für die Gebäude-Ebene auf der Karte.
    \item levelController : MapLayerDataController MapLayerDataController für die Etagen-Ebene auf der Karte.
    \item levelChangeController : LevelChangeControllers LevelChangeControllers für das Wechseln der Etage.
\end{itemize}

\paragraph*{Interaktionen}
Folgende Interaktionen mit der Karte müssen aktiviert beziehungsweise sichergestellt werden:
\begin{itemize}
    \item Der Benutzer kann in die Karte durch das Zusammen-(bez. Auseinander-)führen von zwei 
    Fingern in die Karte heraus-(bez. hinein)zoomen.
    \item Der Benutzer kann die Karte mit einem Finger verschieben.
    \item Der Benutzer kann die Karte mit zwei Fingern drehen. Dabei muss sich die Kompass-
    Anzeige aktualisieren.
    \item Klickt der Benutzer auf den Button zur Ortung, wird der aktuelle Standort auf der Karte 
    angezeigt. Dazu muss ggf. Standortzugriff angefordert werden (wird dieser verweigert passiert 
    nichts).\\
    Wird der Standort bereits angezeigt, wird die Karte so verschoben, dass sich die eigene Position 
    in der Mitte des Bildschirms befindet. \\
\end{itemize}

\paragraph*{Startposition}
Die Startposition der Karte ist beim Öffnen der App wie folgt definiert:
\begin{itemize}
    \item Ist Standortzugriff aktiviert, wird der eigene Standort auf der Karte mittig angezeigt.
    \item Sind keine Standortinformationen verfügbar, so wird die letzte Position angezeigt 
    (bez. Default-Position, falls diese nicht existiert).
\end{itemize}
Um dieses Verhalten zu ermöglichen, muss dafür gesorgt werden, dass die letzte Position beim Verlassen 
der App gespeichert wird.\\
Ist die App geöffnet und der Benutzer kehrt aus einer anderen Ansicht in die Kartenansicht zurück,
wird die letzte Position angezeigt, sofern nicht eine bestimmte Position angezeigt werden muss 
(siehe Beobachtete Daten: getFocusedPos).

\paragraph*{Beobachtete Daten}
Dieses Composable beobachtet einen MapDisplayDataProvider. 
Eine Änderung der Daten wirkt sich wie folgt aus:
\begin{itemize}
    \item getTileSource: Die Karte, die von der MapView angezeigt wird, wird mit den neuen 
    Kartendaten aktualisiert.
    \item getFocusedPos: Die neue fokussierte Position wird mittig auf der Karte angezeigt.
\end{itemize}

\paragraph*{Notizen zur Implementierung}
Zur Implementierung können die Klassen \href{https://osmdroid.github.io/osmdroid/javadocs/osmdroid-android/debug/org/osmdroid/views/overlay/mylocation/MyLocationNewOverlay.html}
{MyLocationNewOverlay} für das Anzeigen des Standorts und \href{https://osmdroid.github.io/osmdroid/javadocs/osmdroid-android/debug/org/osmdroid/views/overlay/compass/CompassOverlay.html}
{CompassOverlay} für das Anzeigen des Kompasses in Betracht gezogen werden.