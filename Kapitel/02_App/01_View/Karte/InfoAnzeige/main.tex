\section{MapObjectInfoDisplay}
\paragraph*{Typ}
Composable.
\paragraph*{Beschreibung}
Dieses Composable zeigt Informationen zum ausgewählten Kartenobjekt an.\\
Es gibt zwei verschiedene Anzeigemodi:

\subparagraph*{Kompakte Anzeige}
Wird das MapObjectInfoDisplay geöffnet, wird es nur in der unteren Bildschirmhälfte angezeigt. 
Es werden nur die wichtigsten Informationen angezeigt. Es wird nach Typ des Kartenobjekts unterschieden:
\begin{itemize}
    \item Gebäude: Es werden nur Gebäudenummer und Adresse angezeigt.
    \item Raum: Es werden Raumnummer, wenn vorhanden Raumtyp, Gebäudenummer des zugehörigen Gebäudes und Stockwerk angezeigt.
\end{itemize}
Rechts unten wird ein Button zum Wechsel in die ausführliche Anzeige angezeigt.

\subparagraph*{Ausführliche Anzeige}
Die ausführliche Anzeige bedeckt den ganzen Bildschirm.
Es wird folgendes angezeigt:
\begin{itemize}
    \item Informationen der kompakten Anzeige.
    \item Wenn es sich um einen Raum handelt: Alle zugehörigen Personen. 
    Existiert diese Information nicht, wird diese Kategorie nicht angezeigt.
    \item Alias-Vorschläge: Es werden mehrere Alias-Vorschläge in je einer Zeile angezeigt. 
    Am Ende der Zeile befinden sich ein Button mit einem Daumen nach oben und ein Button mit einem Daumen nach unten.\\
    Gibt es keine Alias-Vorschläge wird diese Kategorie nicht angezeigt. 
    Wird auf einen dieser Buttons geklickt, wird dies dem ViewModel mitgeteilt.
    \item Globale Aliasse: Es werden globale Aliasse für dieses Kartenobjekt angezeigt.
    \begin{itemize}
        \item Gibt es sehr viele globale Aliasse, sodass die Kategorie über 3 Zeilen lang ist, werden nur 
        die ersten 3 Zeilen angezeigt. Rechts unten befindet sich dann ein Button \dq{}mehr anzeigen\dq{}. 
        Wird auf diesen Button geklickt werden an gleicher Stelle alle globalen Aliasse angezeigt. 
        Der Button enthält dann den Text \dq{}weniger anzeigen\dq{}. Wird er nun geklickt wird wieder nur die 
        reduzierte Anzahl globaler Aliasse angezeigt und der Button zeigt wieder \dq{}mehr anzeigen\dq{}.
        \item Gibt es noch keine globalen Aliasse wird ein Text angezeigt. Dieser Text weist den Benutzer 
        darauf hin, dass es noch keine globalen Aliasse gibt und dass der Benutzer selbst einen vorschlagen kann.
    \end{itemize}
    Neben der Überschrift befindet sich auf der rechten Seite ein Button, auf dem \dq{}Vorschlagen\dq{} oder ein gleichbedeutendes Symbol steht.
    Wird auf diesen Button geklickt, wird dies dem ViewModel mitgeteilt.
    \item Lokale Aliasse: Es werden lokale Aliasse für dieses Kartenobjekt angezeigt. 
    \begin{itemize}
        \item Gibt es sehr viele lokale Aliasse, sodass die Kategorie über 3 Zeilen lang ist, werden nur 
        die ersten 3 Zeilen angezeigt. Rechts unten befindet sich dann ein Button \dq{}mehr anzeigen\dq{}. 
        Wird auf diesen Button geklickt, werden an gleicher Stelle alle lokalen Aliasse angezeigt. 
        Der Button enthält dann den Text \dq{}weniger anzeigen\dq{}. Wird er nun geklickt, wird wieder nur die 
        reduzierte Anzahl lokaler Aliasse angezeigt und der Button zeigt wieder \dq{}mehr anzeigen\dq{}.
        \item Gibt es noch keine lokalen Aliasse wird ein Text angezeigt. Dieser Text weist den Benutzer 
        darauf hin, dass es noch keine lokalen Aliasse gibt und dass der Benutzer selbst einen hinzufügen kann.
        \item Gibt es mindestens einen lokalen Alias, so wird unter den Aliassen ein Text angezeigt, 
        der den Benutzer darauf hinweist, dass er durch langes Klicken auf einen lokalen Alias, diesen 
        löschen kann.\\
        Klickt der Benutzer auf einen der angezeigten lokalen Aliasse und hält diesen für mindestens eine 
        Sekunde gedrückt, erscheint eine Meldung, ob der Benutzer den ausgewählten lokalen Alias wirklich 
        löschen möchte. Es gibt die Auswahlmöglichkeiten \dq{}Abbrechen\dq{} und \dq{}Bestätigen\dq{}.\\ 
        Wird auf \dq{}Bestätigen\dq{} geklickt, wird dies dem ViewModel mitgeteilt und die Meldung schließt sich.\\
        Wird auf \dq{}Abbrechen\dq{} geklickt, oder die Zurück-Taste (des Endgeräts) betätigt, schließt sich die Meldung ebenfalls.
    \end{itemize}
    Neben der Überschrift befindet sich auf der rechten Seite ein Button auf dem \dq{}Hinzufügen\dq{} oder ein gleichbedeutendes Symbol steht. 
    Wird auf diesen geklickt, wechselt die App in den Screen zum Hinzufügen eines lokalen Aliasses.
\end{itemize}
Kann der gesamte Inhalt der ausführlichen Anzeige nicht auf einem Bildschirm angezeigt werden, erlaubt sie es zu scrollen.

\paragraph*{Wechsel zwischen Anzeigen}
\subparagraph*{Wechsel von kompakter in ausführliche Anzeige}
Es gibt zwei Möglichkeiten, von der kompakten in die ausführliche Anzeige zu wechseln:
\begin{itemize}
    \item Der Benutzer klickt auf den Button zum Wechsel in die ausführliche Anzeige rechts unten.
    \item Der Benutzer packt die kompakte Anzeige im oberen Teil (Überschrift und höher) und zieht sie nach oben (drag-Operation).
\end{itemize}
In beiden Fällen gibt es die gleiche Animation beim Wechseln: Die kompakte Anzeige wird nach oben hin immer 
höher, bis sie so hoch ist wie der Bildschirm. Danach befindet sie sich in der ausführlichen Anzeige.

\subparagraph*{Wechsel von ausführlicher zurück in kompakte Anzeige}
Es gibt drei Möglichkeiten, von der ausführlichen in die kompakte Anzeige zu wechseln:
\begin{itemize}
    \item Der Benutzer betätigt die Zurück-Taste (des Endgeräts).
    \item Der Benutzer packt die ausführliche Anzeige im oberen Teil (Überschrift und höher) und zieht sie nach unten (drag-Operation).
    \item Der Benutzer klickt auf den Button mit einem Pfeil nach unten, der neben der Überschrift angezeigt wird.
\end{itemize}
In allen Fällen gibt es die gleiche Animation beim Wechseln: Die ausführliche Anzeige wird immer 
kleiner, bis sie so hoch ist wie die kompakte Anzeige. Danach befindet sie sich in der kompakten Anzeige.

\subparagraph*{Schließen der kompakte Anzeige}
Es gibt zwei Möglichkeiten die kompakte Anzeige zu schließen:
\begin{itemize}
    \item Der Benutzer betätigt die Zurück-Taste (des Endgeräts).
    \item Der Benutzer packt die kompakte Anzeige im oberen Teil (Überschrift und höher) und zieht sie nach unten (drag-Operation). 
\end{itemize}
In beiden Fällen gibt es die gleiche Animation: Die kompakte Anzeige verschwindet nach unten. Danach ist sie geschlossen.

\paragraph*{Parameter}
\begin{itemize}
    \item controller : IMapObjectInfoController IMapObjectInfoController von dem diese Klasse ihre Daten bezieht.
\end{itemize}
