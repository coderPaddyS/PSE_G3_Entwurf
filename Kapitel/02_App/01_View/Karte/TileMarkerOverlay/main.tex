\section{TileMarkerOverlay}
\paragraph*{Typ} 
Abstrakte Klasse, erbt von Overlay.
\paragraph*{Entwurfsmuster}
\subparagraph*{Dekorierer}
Diese Klasse ist ein leicht abgewandelter Dekorierer von Overlay.
Die Abwandlung besteht darin, dass das Objekt, an das delegiert wird nicht ein
beliebiges Overlay, sondern ein FolderOverlay ist. Außerdem werden die neuen 
Methoden direkt in dieser Klasse hinzugefügt.\\
\subparagraph*{Beobachter}
Diese Klasse ist ein Beobachter. Das beobachtete Subjekt ist hier ein MapLayerDataController.
Die Beobachtung geschieht über die LiveData Mechanismen.
\paragraph*{Beschreibung}
Dieses Overlay besitzt ein TileOverlay (also eine Kartenebene) und eine 
Liste von MapObjectMarkern. Diese werden über dem TileOverlay angezeigt.
Für die Verwaltung dieser Overlays wird das FolderOverlay eingesetzt.\\
Diese Klasse verwaltet die MapObjectMarker, die Aktion beim Klicken auf diese Marker, 
und das Fokussieren eines dieser Marker.

\subsubsection{\texttt{+ <<create>> TileMarkerOverlay(controller : MapLayerDataController)}}%$$$M
\paragraph*{Kurzbeschreibung}
Erstellt eine TileMarkerOverlay-Instanz.
\paragraph*{Beschreibung}
Beginnt die von controller bereitgestellten Daten zu beobachten. Erzeugt das TileOverlay.
Erzeugt die MapObjectMarker über die Fabrikmethode createMarker.
\paragraph*{Parameter}
\begin{itemize}
    \item controller : MapLayerDataController MapLayerDataController der die Daten bereitstellt.
\end{itemize}

\subsection{Delegierte Methoden}
Für das Dekorierer-Entwurfsmuster werden folgende Methoden von Overlay ohne Veränderung 
an die FolderOverlay-Instanz delegiert:
\begin{itemize}
    \item draw(pCanvas:Canvas, pProjection:Projection)
    \item draw(pCanvas:Canvas, pMapView:MapView, pShadow:boolean)
    \item onSingleTapUp(e:MotionEvent, mapView:MapView):boolean
    \item onSingleTapConfirmed(e:MotionEvent, mapView:MapView):boolean
    \item onLongPress(e:MotionEvent, mapView:MapView):boolean
    \item onTouchEvent(e:MotionEvent, mapView:MapView):boolean
    \item onDetach(mapView:MapView)
\end{itemize}

\subsubsection{\texttt{+ onMarkerTap(mapId : int)}}%$$$M
\paragraph*{Beschreibung}
Teilt dem MapLayerDataController mit, dass ein bestimmtes Kartenobjekt ausgewählt wurde.
\paragraph*{Parameter}
\begin{itemize}
    \item mapId : int MapId des Kartenobjekts, auf das ausgewählt wurde.
\end{itemize}
\paragraph*{Rückgabewert}
Void.

\subsubsection{\texttt{# createMarker(data : DisplayData) : MapObjectMarker $\lbrace$abstract$\rbrace$}}%$$$M
\paragraph*{Beschreibung}
Dies ist eine Fabrikmethode zur Erzeugung eines MapObjectMarker.
\paragraph*{Parameter}
\begin{itemize}
    \item data : DisplayData DisplayData mit den Anzeige-Daten für den Marker.
    \item mapView : MapView MapView auf der der Marker angezeigt wird.
\end{itemize}
\paragraph*{Rückgabewert}
MapObjectMarker.


\subsection{Weitere private Funktionalität des Beobachters}
Ändern sich die beobachteten Daten aktualisieren sich die Overlays wie folgt:
\begin{itemize}
    \item TileSource: Das TileOverlay wird mit den neuen Kartendaten aktualisiert.
    Ist TileSource null, wird gar nichts angezeigt (also auch keine Marker).
    \item MapObjectData: Alle Marker werden verworfen und mit den neuen Daten neu erzeugt (über createMarker).
    \item FocusedMapId und FloorFocus: Der aktuell fokussierte Marker wird zurückgesetzt.
    Dem Marker mit der MapId FocusedMapId wird über seine focusMarkerObject-Methode FloorFocus übergeben.
\end{itemize}

%-------------------------------------------------------------
\section{BuildingOverlay}
\paragraph*{Typ}
Erbt von TileMarkerOverlay.
\paragraph*{Beschreibung}
Diese Klasse ist ein TileMarkerOverlay für die Gebäudeebene auf der Karte.

\subsubsection{\texttt{# createMarker(data : DisplayData) : MapObjectMarker}}%$$$M
\paragraph*{Kurzbeschreibung}
Erzeugt einen BuildingMarker mit den gegeben Daten.
\paragraph*{Beschreibung}
Die Fabrikmethode zur Erzeugung eines Markers wird hier mit einem BuildingMarker implementiert.
\paragraph*{Parameter}
\begin{itemize}
    \item data : DisplayData DisplayData mit den Anzeige-Daten für den Marker.
    \item mapView : MapView MapView auf der der Marker angezeigt wird.
\end{itemize}
\paragraph*{Rückgabewert}
MapObjectMarker, erzeugter MapObjectMarker.

\subsubsection{\texttt{Weitere Implementierungen}}%$$$M
Benutzt den von TileMarkerOverlay geerbten Konstruktor.

%-------------------------------------------------------------
\section{FloorOverlay}
\paragraph*{Typ}
Erbt von TileMarkerOverlay.
\paragraph*{Beschreibung}
Diese Klasse ist ein TileMarkerOverlay für die Etagenebene auf der Karte.

\subsubsection{\texttt{# createMarker(data : DisplayData) : MapObjectMarker}}%$$$M
\paragraph*{Kurzbeschreibung}
Erzeugt einen RoomMarker mit den gegeben Daten.
\paragraph*{Beschreibung}
Die Fabrikmethode zur Erzeugung eines Markers wird hier mit einem RoomMarker implementiert.
\paragraph*{Parameter}
\begin{itemize}
    \item data : DisplayData DisplayData mit den Anzeige-Daten für den Marker.
    \item mapView : MapView MapView auf der der Marker angezeigt wird.
\end{itemize}
\paragraph*{Rückgabewert}
MapObjectMarker, erzeugter MapObjectMarker.

\subsubsection{Weitere Implementierungen}%$$$M
Benutzt den von TileMarkerOverlay geerbten Konstruktor.