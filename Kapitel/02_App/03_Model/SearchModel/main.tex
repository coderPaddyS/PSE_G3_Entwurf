\subsection{Interface ISearchModel}
\paragraph*{Beschreibung}
Dieses Interface ermöglicht das Durchsuchen der Datenbank nach gültigen Suchbegriffen zu einem gegebenen String.
Es ermöglicht die Bestimmung der MapId zu einem gültigen Suchbegriff, 
dieser kann eine Gebäudenummer, Gebäudenummer und Raumnummer, Gebäudenummer und Alias eines Raumes, ein Alias eines Gebäudes oder eine Person sein.
Es ermöglicht die Speicherung und Ausgabe von zuletzt gesuchten Suchbegriffen.

\subsubsection{\texttt{+ getSuggestions(input : String, amount : int) : List<String>}}%$$$M
\paragraph*{Kurzbeschreibung}
Diese Methode gibt Suchvorschläge zu dem übergebenen String zurück.
\paragraph*{Beschreibung}
Diese Methode gibt Suchvorschläge zu dem übergebenen String aus einer (SQL-)Datenbank, 
die alle Kartenobjekte inklusive ihrer offiziellen Namen, Aliasse und Personen enthält, zurück.
\paragraph*{Parameter}
\begin{itemize}
    \item input : String = Der Suchbegriff den der Benutzer eingegeben hat
    \item amout : int = Die maximale Anzahl der Elemente der Liste an Suchvorschlägen
\end{itemize}
\paragraph*{Rückgabewert}
List<String>, Suchvorschläge, die die Datenbank zurückgegeben hat
\paragraph*{Notizen zur Implementierung}
Es werden die Suchvorschläge nach der Übereinstimmung mit der Eingabe sortiert zurück gegeben.
Die Suchvorschläge mit der höchsten Übereinstimmung sind die ersten Elemente der Liste.
Die Liste enthält genau so viele Vorschläge, wie die Anzahl vorgibt. Die Anzahl ist nicht negativ.
Somit werden auch Vorschläge für falsch geschriebene Suchbegriffe angezeigt.
Wenn SQL genutzt wird, wird die Suche direkt auf SQL ausgeführt z.B. mit \dq like\dq{}.

\subsubsection{\texttt{+ getLastSearches(amount : int) : List<String>}}%$$$M
\paragraph*{Kurzbeschreibung}
Diese Methode gibt eine Liste der zuletzt gesuchten Suchbegriffe zurück.
\paragraph*{Beschreibung}
Diese Methode gibt eine Liste der zuletzt gesuchten Suchbegriffe, nach ihrer letzten Benutzung sortiert, aus der Datenbank zurück.
\paragraph*{Parameter}
\begin{itemize}
    \item amout : int = Die maximale Anzahl der Elemente der Liste an letzten Suchbegriffen
\end{itemize}
\paragraph*{Rückgabewert}
List<String>, zuletzt gesuchte Suchbegriffe
\paragraph*{Notizen zur Implementierung}
Die Anzahl der Elemente ist die übergebene Anzahl, es sei den es gibt nicht genug bereits gesuchte Suchbegriffe.
Der zuletzt genutzte Suchbegriff ist das erste Element der Liste.

\subsubsection{\texttt{+ storeSearchKey(searchKey : String)}}%$$$M
\paragraph*{Kurzbeschreibung}
Diese Methode speichert den gegebenen Suchbegriff.
\paragraph*{Beschreibung}
Diese Methode speichert den gegebenen Suchbegriff in der Datenbank als zuletzt gesucht.
\paragraph*{Parameter}
\begin{itemize}
    \item searchKey : String = Suchbegriff, der gespeichert werden soll
\end{itemize}
\paragraph*{Rückgabewert}
keinen
\paragraph*{Notizen zur Implementierung}
Wenn der Suchbegriff bereits in der Liste der zuletzt gesuchten Suchbegriffe ist, 
wird der Suchbegriff zuerst aus der Liste entfernt und dann als neuster zuletzt gesuchter Suchbegriff hinzugefügt.
Somit sind die bereits gesuchten Suchbegriffe immer danach sortiert, wann sie zuletzt gesucht wurden.

\subsubsection{\texttt{+ getMapIdBySearchKey(searchKey : String) : int}}%$$$M
\paragraph*{Kurzbeschreibung}
Diese Methode gibt die MapId des zum Suchbegriff gehörenden Kartenobjekts zurück.
\paragraph*{Beschreibung}
Diese Klasse ermittelt das zu den Suchbegriffen gehörenden Kartenobjekt in der Datenbank und gibt die MapId dieses zurück.
\paragraph*{Parameter}
\begin{itemize}
    \item searchKey : String = Suchbegriff, dessen MapId bestimmt werden soll
\end{itemize}
\paragraph*{Rückgabewert}
int, MapId zum Suchbegriff


\subsection{class SearchModel implements ISearchModel}
\paragraph*{Beschreibung}
Klasse zur Verwaltung der Daten zum Suchen von Kartenobjekten mit Hilfe eines Strings: 
Bestimmung von Suchvorschlägen, Speicherung und Ausgabe von zuletzt gesuchten Suchbegriffen, Bestimmung der MapId zu Suchbegriffen.