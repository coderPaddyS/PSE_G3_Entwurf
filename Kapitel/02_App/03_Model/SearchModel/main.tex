\subsection{SearchModel}
\paragraph*{Beschreibung}
Diese Klasse ermöglicht das Durchsuchen der Datenbank nach gültigen Suchbegriffen zu der gegebenen Query 
und ermöglicht die Bestimmung des Kartenobjekts zu einem gültigen Suchbegriff, 
dies kann eine Gebäudenummer, Gebäudenummer und Raumnummer, Gebäudenummer und Alias eines Raumes, ein Alias eines Gebäudes oder eine Person sein.

\subsubsection{\texttt{+ getSuggestions(query : Query) : List<String>}}%$$$M
\paragraph*{Kurzbeschreibung}
Diese Methode gibt Suchvorschläge zu der übergebenen Query zurück.
\paragraph*{Beschreibung}
Diese Methode gibt Suchvorschläge zu der übergebenen Query aus einer (SQL-)Datenbank, 
die alle Kartenobjekte inklusive ihrer offiziellen Namen, Aliasse und Personen enthält, zurück.
Wenn möglich sollen auch Rechtschreibfehler einkalkuliert werden, so dass auch Vorschläge für falsch geschriebene Suchbegriffe kommen.
\paragraph*{Parameter}
\begin{itemize}
    \item query : Query Der Suchbegriff den der Benutzer eingegeben hat
\end{itemize}
\paragraph*{Rückgabewert}
List<String>, Suchvorschläge, die die Datenbank zurückgegeben hat
\paragraph*{Notizen zur Implementierung}
Wenn SQL genutzt wird, wird die Suche direkt auf SQL ausgeführt z.B. mit \dq like\dq{}.

\subsubsection{\texttt{+ getSearchable(searchKey : String) : MapObject}}%$$$M
\paragraph*{Kurzbeschreibung}
Diese Methode gibt das zum Suchbegriff gehörenden Kartenobjekt zurück.
\paragraph*{Beschreibung}
Diese Klasse ermittelt das zu den Suchbegriffen gehörenden Kartenobjekt in der Datenbank und gibt dieses zurück.
\paragraph*{Parameter}
\begin{itemize}
    \item searchKey : String Suchbegriff
\end{itemize}
\paragraph*{Rückgabewert}
MapObject, Kartenobjekt zum Suchbegriff

\subsubsection{\texttt{+ storeSearchKey(searchKey : String)}}%$$$M
\paragraph*{Kurzbeschreibung}
Diese Methode speichert den gegebenen Suchbegriff.
\paragraph*{Beschreibung}
Diese Methode speichert den gegebenen Suchbegriff in der Datenbank als zuletzt gesucht.
\paragraph*{Parameter}
\begin{itemize}
    \item searchKey : String
\end{itemize}
\paragraph*{Rückgabewert}
keinen

\subsubsection{\texttt{+ getLastSearchKeys() : List<String>}}%$$$M
\paragraph*{Kurzbeschreibung}
Diese Methode gibt die zuletzt gesuchten Suchbegriffe zurück.
\paragraph*{Beschreibung}
Diese Methode gibt die zuletzt gesuchten Suchbegriffe aus der Datenbank und gibt diese zurück.
\paragraph*{Parameter}
\begin{itemize}
    \item keine
\end{itemize}
\paragraph*{Rückgabewert}
List<String>, zuletzt gesuchte Suchbegriffe