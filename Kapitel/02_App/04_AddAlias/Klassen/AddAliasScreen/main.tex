\subsubsection{AddAliasScreen}
\paragraph*{Typ}
Composable.
\paragraph*{Beschreibung}
Dieses Composable stellt eine grafische Benutzerschnittstelle bereit.
Mit dieser kann der Benutzer einen neuen Alias lokal hinzufügen oder global vorschlagen.\\
Beim Erstellen eines AddAliasScreens durch die Navigation erhält dieser die MapId 
des ausgewählten Kartenobjekts, zu dem ein Alias erstellt werden soll (siehe Navigation).\\
Für die Speicherung der neuen Aliasse bez. Alias-Vorschläge und den Erhalt der Aliasse benutzt 
der AddAliasScreen einen IAddAliasController.

\paragraph*{Aufbau}
\begin{itemize}
    \item In der Kopfzeile befindet sich ein Zurück-Button. Wird auf diesen geklickt, oder die Zurück-Funktion des Endgeräts benutzt, 
    kehrt die App zur vorherigen Ansicht zurück. Eingetippte Aliasse, die noch nicht gespeichert oder vorgeschlagen wurden verfallen.
    \item Es wird die Gebäudenummer des ausgewählten Gebäudes, bez. die des Gebäudes, in dem sich der Raum befindet angezeigt.
    \item Ist das ausgewählte Kartenobjekt ein Raum, wird unter der Gebäudenummer ebenfalls die Raumnummer angezeigt.
    \item Darunter wird ein Text angezeigt, der den Benutzer dazu auffordert einen neuen Alias einzugeben.
    \item Darunter befindet sich ein Textfeld. Klickt der Benutzer auf dieses Textfeld, öffnet sich die Tastatur. 
    Der Benutzer kann einen Text in das Textfeld eingeben.
    \item Darunter befinden sich nebeneinander zwei Buttons. Einer zum lokalen Speichern und einer zum globalen Vorschlagen. 
    Beim Klick auf einen der beiden Buttons wird die jeweilige Aktion ausgeführt. War die Aktion erfolgreich, wird die Eingabe 
    aus dem Textfeld gelöscht. Der Screen zeigt in jedem Fall eine Meldung zum Erfolg bez. Misserfolg der Aktion an.
    \item Darunter werden für dieses Kartenobjekt bereits existierende globale Aliasse angezeigt.
\end{itemize}
