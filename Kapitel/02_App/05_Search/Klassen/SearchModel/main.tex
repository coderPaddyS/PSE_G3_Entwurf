\subsubsection{ISearchModel}\label{App_Search_ISearchModel}
\paragraph*{Typ}
\texttt{Interface}.
\paragraph*{Beschreibung}
Dieses Interface ermöglicht das Durchsuchen der Datenbank nach gültigen Suchbegriffen zu einem gegebenen String.
Es ermöglicht die Bestimmung der MapId zu einem gültigen Suchbegriff, 
dieser kann eine Gebäudenummer, Gebäudenummer und Raumnummer, Gebäudenummer und Alias eines Raumes, ein Alias eines Gebäudes oder eine Person sein.
Es ermöglicht die Speicherung und Ausgabe von zuletzt gesuchten Suchbegriffen.

 \subsubsection*{\texttt{+ getSuggestions(input : String, amount : int) : List<String>}}\label{App_Search_ISearchModel_getSuggestions}%$$$M
\paragraph*{Kurzbeschreibung}
Gibt Suchvorschläge zu dem übergebenen String zurück.
\paragraph*{Beschreibung}
Diese Methode gibt Suchvorschläge zu dem übergebenen String aus einer Datenbank, 
die alle Kartenobjekte inklusive ihrer offiziellen Namen, Aliasse und Personen enthält, zurück.
Die Suche wird direkt auf der Datenbank ausgeführt, wie z.B. mit \dq like \dq{} von SQL.
Es werden die Suchvorschläge nach der Übereinstimmung mit der Eingabe sortiert zurück gegeben.
Die Suchvorschläge mit der höchsten Übereinstimmung sind die ersten Elemente der Liste.
Somit werden auch Vorschläge für falsch geschriebene Suchbegriffe angezeigt.
\paragraph*{Parameter}
\begin{itemize}
    \item \texttt{input : String} Der Suchbegriff den der Benutzer eingegeben hat.
    \item \texttt{amount : int} Die maximale Anzahl der Elemente der Liste an Suchvorschlägen, nicht negativ.
\end{itemize}
\paragraph*{Rückgabewert}
\texttt{List<String>}, Suchvorschläge, die die Datenbank zurückgegeben hat.

 \subsubsection*{\texttt{+ getLastSearches(amount : int) : List<String>}}\label{App_Search_ISearchModel_getLastSearches}%$$$M
\paragraph*{Kurzbeschreibung}
Gibt eine Liste der zuletzt gesuchten Suchbegriffe zurück.
\paragraph*{Beschreibung}
Diese Methode gibt eine Liste der zuletzt gesuchten Suchbegriffe, nach ihrer letzten Benutzung sortiert (zuletzt gesuchten zuerst), zurück.
\paragraph*{Parameter}
\begin{itemize}
    \item \texttt{amount : int} Die maximale Anzahl der Elemente der Liste an letzten Suchbegriffen.
\end{itemize}
\paragraph*{Rückgabewert}
\texttt{List<String>}, zuletzt gesuchte Suchbegriffe.

 \subsubsection*{\texttt{+ storeSearchKey(searchKey : String)}}\label{App_Search_ISearchModel_storeSearchKey}%$$$M
\paragraph*{Beschreibung}
Speichert den gegebenen Suchbegriff mit dem aktuellen Datum im Suchverlauf.
\paragraph*{Parameter}
\begin{itemize}
    \item \texttt{searchKey : String} Suchbegriff, der gespeichert werden soll.
\end{itemize}
\paragraph*{Rückgabewert}
\texttt{Void}.

 \subsubsection*{\texttt{+ getMapIdBySearchKey(searchKey : String) : int}}\label{App_Search_ISearchModel_getMapIdBySearchKey}%$$$M
\paragraph*{Kurzbeschreibung}
Gibt die MapId des zum Suchbegriff gehörenden Kartenobjekts zurück.
\paragraph*{Beschreibung}
Diese Methode ermittelt das zu den Suchbegriffen gehörenden Kartenobjekt in der Datenbank und gibt die MapId dieses zurück.
\paragraph*{Parameter}
\begin{itemize}
    \item \texttt{searchKey : String} Suchbegriff, dessen MapId bestimmt werden soll.
\end{itemize}
\paragraph*{Rückgabewert}
\texttt{int}, MapId zum Suchbegriff.


\subsubsection{SearchModel}\label{App_Search_SearchModel}
\paragraph*{Typ}
Implementiert \texttt{ISearchModel}.
\paragraph*{Beschreibung}
Klasse zur Verwaltung der Daten zum Suchen von Kartenobjekten mit Hilfe eines Strings: 
Bestimmung von Suchvorschlägen, Speicherung und Ausgabe von zuletzt gesuchten Suchbegriffen, Bestimmung der MapId zu Suchbegriffen.