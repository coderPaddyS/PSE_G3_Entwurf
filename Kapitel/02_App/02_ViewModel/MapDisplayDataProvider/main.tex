\section{MapDisplayDataProvider}
\paragraph*{Typ}
Interface.
\paragraph*{Beschreibung}
Dieses Interface ist eine Schnittstelle für Lade- und Speicheroperationen zur angezeigten Karte.\\
Klassen dieses Interfaces können von einem Beobachter beobachtet werden. Dazu werden 
LiveData-Klassen benutzt.

\subsubsection{+ getTileSource() : LiveData<ITileSource>}%$$$M
\paragraph*{Beschreibung}
Gibt die ITileSource mit den Kartendaten zurück.
\paragraph*{Parameter}
Keine.
\paragraph*{Rückgabewert}
LiveData<TileSource> mit den Kartendaten.

\subsubsection{+ saveLastPos(pos : IGeoPoint)}%$$$M
\paragraph*{Beschreibung}
Speichert eine gegebene Position als letzte Position.
\paragraph*{Parameter}
\begin{itemize}
    \item pos:IGeoPoint zu speichernde Position.
\end{itemize}
\paragraph*{Rückgabewert}
Void.

\subsubsection{+ getLastPos() : IGeoPoint}%$$$M
\paragraph*{Kurzbeschreibung}
Gibt die letzte Position zurück.
\paragraph*{Beschreibung}
Diese Methode gibt die als letzte Position gespeicherte Position als IGeoPoint zurück.
Wurde noch keine Position gespeichert, wird eine Defaut-Position zurückgegeben.
Dies wird benötigt, damit die Karte beim Neuladen die gleiche Stelle zeigen kann.
\paragraph*{Parameter}
Keine.
\paragraph*{Rückgabewert}
IGeoPoint, letzte Position.

\subsubsection{+ getFocusedPos() : LiveData<IGeoPoint>}%$$$M
\paragraph*{Kurzbeschreibung}
Gibt die Position zurück, die mittig auf der Karte angezeigt werden muss.
\paragraph*{Beschreibung}
Wird ein Objekt auf der Karte fokussiert, und die Karte muss diese Position mittig auf dem 
Bildschirm anzeigen (z.B. um ein Suchergebnis anzuzeigen), so wird dieser IGeoPoint aktualisiert.\\
Darf null sein, wenn keine bestimmte Position fokussiert werden muss.
\paragraph*{Parameter}
Keine.
\paragraph*{Rückgabewert}
LiveData<IGeoPoint>, Position, die mittig angezeigt werden soll.
