\subsection{Interface ISearchController}
\paragraph*{Beschreibung}
Dieses Interface ermöglicht die Suche nach Gebäuden und Räumen zu der gegebenen Query 
und ermöglicht die Bestimmung des Kartenobjekts zu einem gültigen Suchbegriff, 
dies kann eine Gebäudenummer, Gebäudenummer und Raumnummer, Gebäudenummer und Alias eines Raumes, ein Alias eines Gebäudes oder eine Person sein.
Dies macht sie, indem sie die Daten bei SearchModel anfordert.


\subsubsection{\texttt{+ getSuggestions() : LiveData<List<String>>}}%$$$M
\paragraph*{Kurzbeschreibung}
Gibt Suchvorschläge zu dem in einer Variablen gespeicherten Input zurück.
\paragraph*{Beschreibung}
Fordert Suchvorschläge zu dem in einer Variablen gespeicherten Input von einer Klasse, die ISearchModel implementiert, an und gibt diese zurück.
Wenn der Input leer ist werden stattdessen die letzten Suchbegriffe angefordert und zurückgegeben.
\paragraph*{Parameter}
\begin{itemize}
    \item keine
\end{itemize}
\paragraph*{Rückgabewert}
LiveData<List<String>>, Suchvorschläge zum Input, die eine Klasse, die ISearchModel implementiert, zurückgegeben hat

\subsubsection{\texttt{+ setInput(input : String)}}%$$$M
\paragraph*{Kurzbeschreibung}
Speichert den gegebenen Input in einer Variablen.
\paragraph*{Beschreibung}
Speichert den gegebenen Input in einer Variablen, indem der alte Wert der Variablen überschrieben wird.
\paragraph*{Parameter}
\begin{itemize}
    \item input : String = Der Input, den der Benutzer zur Suche eingegeben hat
\end{itemize}
\paragraph*{Rückgabewert}
void

\subsubsection{\texttt{+ confirminput()}}%$$$M
\paragraph*{Kurzbeschreibung}
Zeigt das Kartenobjekts zu dem obersten angezeigten Suchvorschlag an.
\paragraph*{Parameter}
Fordert die Speicherung des obersten angezeigten Suchvorschlags von einer Klasse, die ISearchModel implementiert.
Fordert die MapId des obersten angezeigten Suchvorschlags von einer Klasse, die ISearchModel implementiert, an.
Fordert von dem Navigationscontroller, das er die Karte mit der MapId fokussiert läd.
\begin{itemize}
    \item keine
\end{itemize}
\paragraph*{Rückgabewert}
void

\subsubsection{\texttt{+ selectSuggestion(suggestion : String)}}%$$$M
\paragraph*{Kurzbeschreibung}
Zeigt das Kartenobjekts zu dem ausgewählten Suchvorschlag an.
\paragraph*{Parameter}
Fordert die Speicherung des ausgewählten Suchvorschlags von einer Klasse, die ISearchModel implementiert.
Fordert die MapId des ausgewählten Suchvorschlags von einer Klasse, die ISearchModel implementiert, an.
Fordert von dem Navigationscontroller, das er die Karte mit der MapId fokussiert läd.
\begin{itemize}
    \item suggestion : String = Der ausgewählte Suchvorschlag
\end{itemize}
\paragraph*{Rückgabewert}
void

\subsection{class SearchController implements ISearchController}