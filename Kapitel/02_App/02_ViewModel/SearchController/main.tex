\subsection{Search}
\paragraph*{Beschreibung}
Diese Klasse ermöglicht die Suche nach Gebäuden und Räumen zu der gegebenen Query 
und ermöglicht die Bestimmung des Kartenobjekts zu einem gültigen Suchbegriff, 
dies kann eine Gebäudenummer, Gebäudenummer und Raumnummer, Gebäudenummer und Alias eines Raumes, ein Alias eines Gebäudes oder eine Person sein.
Dies macht sie, indem sie die Daten bei SearchModel anfordert.

\subsubsection{\texttt{+ getSearchInput() : String}}%$$$M
\paragraph*{Kurzbeschreibung}
Diese Methode ermöglicht Benutzereingaben und gibt einen gültigen Suchbegriff oder Null zurück.
\paragraph*{Beschreibung}
Diese Methode ermöglicht die Suche von Gebäuden und Räumen über die Eingabe eines Suchbegriffs in ein Textfeld.
Zeigt eine Liste von Vorschlägen (die Suchbegriffe), falls vorhanden, und erlaubt dem Benutzer, einen Vorschlag auszuwählen oder die Eingabe vollständig selbst vorzunehmen.
\paragraph*{Parameter}
\begin{itemize}
    \item keine
\end{itemize}
\paragraph*{Rückgabewert}
String, Suchbegriff, den der Benutzer eingegeben hat
\paragraph*{Notizen zur Implementierung}
Benutzt eine Instanz von SearchView (\href{https://developer.android.com/guide/topics/search/search-dialog#SearchableActivity}).
Für das Durchsuchen der Daten wird getSuggestions von Search genutzt.
Die Suchvorschläge werden einem Cursor übergeben.
Dieser wird von einem CursorAdapter adaptiert und die Vorschläge werden so an die Instanz von SearchView gegeben.
Dort werden die Vorschläge dann dem Benutzer angezeigt.

\subsubsection{\texttt{+ getSuggestions(query : Query) : List<String>}}%$$$M
\paragraph*{Kurzbeschreibung}
Diese Methode gibt Suchvorschläge zu der übergebenen Query zurück.
\paragraph*{Beschreibung}
Diese Methode fordert Suchvorschläge zu der übergebenen Query von dem SearchModel an und gibt diese zurück.
\paragraph*{Parameter}
\begin{itemize}
    \item query : Query Der Suchbegriff den der Benutzer eingegeben hat
\end{itemize}
\paragraph*{Rückgabewert}
LiveData<List<String>>, Suchvorschläge zur Query, die SearchModel zurückgegeben hat

\subsubsection{\texttt{+ getSearchable(searchKey : String) : MapObject}}%$$$M
\paragraph*{Kurzbeschreibung}
Diese Methode gibt das zum Suchbegriff gehörenden Kartenobjekt zurück.
\paragraph*{Beschreibung}
Diese Methode fordert das zu den Suchbegriffen gehörenden Kartenobjekt von dem SearchModel an und gibt dieses zurück.
\paragraph*{Parameter}
\begin{itemize}
    \item searchKey : String Suchbegriff
\end{itemize}
\paragraph*{Rückgabewert}
MapObject, Kartenobjekt zum Suchbegriff, die SearchModel zurückgegeben hat