\section{MapLayerDataController}
\paragraph*{Typ} 
Interface
\paragraph*{Beschreibung}
Dieses Interface definiert eine Schnittstelle, die Daten für eine Kartenebene bereitstellt.
Implementierende Klassen können von einem Beobachter beobachtet werden. Dazu werden die 
Daten für die Karte als LiveData-Objekte herausgegeben.\\
Außerdem wird die Änderung dieser Daten spezifiziert, wenn ein Objekt ausgewählt wird, dessen
Daten im MapLayerDataController verwaltet werden.

\subsubsection{\texttt{+ selectMapObject(mapId : int)}}%$$$M
\paragraph*{Kurzbeschreibung}
Diese Methode fokussiert das richtige Element, nachdem ein Kartenobjekt ausgewählt wurde.
\paragraph*{Beschreibung}
Diese Methode wird aufgerufen, wenn ein Kartenobjekt vom Benutzer ausgewählt wird, dessen 
Daten von diesem MapLayerDataController verwaltet werden.\\
Diese Methode kümmert sich um die Fokussierung des richtigen Kartenobjekts. Eventuell muss 
aus oder in die Etagenansicht gewechselt werden.
\paragraph*{Parameter}
\begin{itemize}
    \item mapId : int MapId des ausgewählten Kartenobjekts.
\end{itemize}
\paragraph*{Rückgabewert}
Void.

\subsubsection{\texttt{+ getTileSource() : LiveData<\href{https://osmdroid.github.io/osmdroid/javadocAll/org/osmdroid/tileprovider/tilesource/ITileSource.html}{ITileSource}>}}%$$$M
\paragraph*{Kurzbeschreibung}
Gibt die \href{https://osmdroid.github.io/osmdroid/javadocAll/org/osmdroid/tileprovider/tilesource/ITileSource.html}
{ITileSource} mit den Daten für eine Kartenebene zurück.
\paragraph*{Beschreibung}
Gibt null zurück, wenn gar nichts angezeigt werden soll.
\paragraph*{Parameter}
Keine.
\paragraph*{Rückgabewert}
LiveData<\href{https://osmdroid.github.io/osmdroid/javadocAll/org/osmdroid/tileprovider/tilesource/ITileSource.html}
{ITileSource}> mit den Kartendaten, darf null sein.

\subsubsection{\texttt{+ getMapObjectData() : LiveData<List<DisplayData>>}}%$$$M
\paragraph*{Beschreibung}
Gibt die Anzeige-Daten der Kartenobjekte zurück.
\paragraph*{Parameter}
Keine.
\paragraph*{Rückgabewert}
LiveData<List<DisplayData>> mit den Daten der Kartenobjekte.

\subsubsection{\texttt{+ getFocusedMapId() : LiveData<int>}}%$$$M
\paragraph*{Beschreibung}
Gibt die MapId des fokussierten Kartenobjekts zurück.
\paragraph*{Parameter}
Keine.
\paragraph*{Rückgabewert}
LiveData<int> MapId des fokussierten Kartenobjekts.

\subsubsection{\texttt{+ isLevelMode() : LiveData<boolean>}}%$$$M
\paragraph*{Kurzbeschreibung}
Gibt zurück, ob das fokussierte Kartenobjekt auf einer Etage ist.
\paragraph*{Beschreibung}
Ist wahr, wenn das fokussierte Kartenobjekt ein Kartenobjekt ist, das im Etagenmodus 
angezeigt wird. Sonst wird false zurückgegeben.
\paragraph*{Parameter}
Keine.
\paragraph*{Rückgabewert}
LiveData<boolean> ob das fokussierte Kartenobjekt auf einer Etage ist.