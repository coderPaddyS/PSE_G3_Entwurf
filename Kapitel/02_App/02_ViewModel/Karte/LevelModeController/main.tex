\subsection{LevelModeController}
\paragraph*{Typ}
Implementiert MapLayerDataController, LevelChangeController.
\paragraph*{Beschreibung}
Diese Klasse steuert den Etagenmodus. Sie stellt alle benötigten Daten bereit, um die aktuelle Etage anzuzeigen. 
Sie stellt alle benötigten Aufrufe bereit, um Interaktionen mit der Etagenkarte zu verarbeiten.

\subsubsection{\texttt{+ <<create>> LevelModeController(dataProvider : IMapDataProvider)}}%$$$M
\paragraph*{Beschreibung}
Erstellt einen neuen LevelModeController.
\paragraph*{Parameter}
\begin{itemize}
    \item dataProvider : IMapDataProvider IMapDataProvider von dem diese Klasse die Kartendaten bezieht.
\end{itemize}

\subsubsection{\texttt{+ startLevelMode(mapId : int)}}%$$$M
\paragraph*{Kurzbeschreibung}
Startet den Etagenansicht eines Gebäudes.
\paragraph*{Beschreibung}
Diese Methode startet für die Etagenansicht für ein gegebenes Gebäude.\\
Existieren die Etagenansicht für das Gebäude nicht, passiert nichts.\\
Existiert die Etagenansicht für das Gebäude werden alle durch das MapLayerDataController-Interface beobachtbaren 
Kartendaten zur Etagen-Layer aktualisiert. Insbesondere ist isLevelMode() \texttt{wahr}.\\
Ist dieses Gebäude das erste, von dem die Etagenansicht betrachtet wird, seit der Kartenmodus geöffnet wurde, 
wird das Erdgeschoss geladen.\\
War dieses Gebäude nicht das letzte, von dem die Etagenansicht betrachtet wurde, wird das Erdgeschoss geladen.\\
War dieses Gebäude das letzte, von dem die Etagenansicht in dieser Kartenmodus-Session geladen war, wird die 
Etage geladen, die zuletzt von diesem Gebäude betrachtet wurde.
\paragraph*{Parameter}
\begin{itemize}
    \item mapId : int MapId des Gebäudes.
\end{itemize}
\paragraph*{Rückgabewert}
Void.
