\subsubsection{SignInScreen}\label{App_Signup_SignInScreen}
\paragraph*{Typ}
\texttt{Composable}.
\paragraph*{Beschreibung}
Dieses \texttt{Composable} zeigt die Anmeldung an.\\
Diese Klasse besitzt zum Erhalt der Daten und zur Ausführung der Interaktionen einen ISignInController.

\paragraph*{Aufbau}
Der SignInScreen ist folgendermaßen aufgebaut:
\begin{itemize}
    \item In der Kopfzeile befindet sich links ein Zurück-Button. Wird dieser geklickt oder die Zurückfunktion 
    des Endgeräts betätigt, kehrt die App in die vorherige Ansicht zurück.
    \item Je nach dem, durch die Navigation übergebenen, Parameter wird ein Text angezeigt:
    \begin{itemize}
        \item \texttt{signin/new}: Der Text weist den Benutzer darauf hin, dass er sich hier anmelden kann.
        \item \texttt{signin/funktion}: Der Text weist den Benutzer darauf hin, dass die gewünschte Funktion nur angemeldeten Benutzern zur Verfügung steht 
        und er sich hier anmelden kann.
        \item \texttt{signin/again}: Der Text weist den Benutzer darauf hin, dass seine Anmeldung abgelaufen ist und er sich hier erneut anmelden muss.
    \end{itemize}
    \item Darunter befindet sich ein Button, durch dessen Klicken die Datenschutzerklärung in der Dokumentenanzeige angezeigt wird.
    \item Darunter befindet sich ein Feld, das man durch Klicken abhaken kann. Daneben steht, dass der Benutzer die Datenschutzerklärung akzeptiert.
    \item Darunter befindet sich ein Text, der den Benutzer darauf hinweist, dass er einen der aufgeführten Anmeldedienste auswählen muss.
    \item Darunter werden alle unterstützen Anmeldedienste angezeigt. Hat der Benutzer die Datenschutzerklärung noch nicht akzeptiert, 
    sind sie ausgegraut und ohne Funktion. Hat der Benutzer die Datenschutzerklärung noch akzeptiert, sind sie nicht ausgegraut. 
    Klickt der Benutzer einen an, wird er zur Anmeldung mit diesem Anmeldedienst weitergeleitet.
    Ist die Anmeldung beim Anmeldedienst abgeschlossen, wird in einer Meldung angezeigt, in der angezeigt wird, ob die Anmeldung erfolgreich war.\\
    War die Anmeldung erfolgreich wird diese Ansicht nach bestätigen der Meldung geschlossen.
\end{itemize}

