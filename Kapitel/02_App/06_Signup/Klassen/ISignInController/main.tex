\subsubsection{ISignInController}
\paragraph*{Typ}
Interface.
\paragraph*{Beschreibung}
Stellt Daten für das Anzeigen der Anmeldung bereit und verarbeitet die Interaktionen.

\subsubsection{\texttt{+ getSupportedProviders() : List<String>}}%$$$M
\paragraph*{Beschreibung}
Gibt eine Liste aller unterstützen Anmeldediensten zurück.
\paragraph*{Parameter}
Keine.
\paragraph*{Rückgabewert}
List<String> mit den Namen aller unterstützen Anmeldediensten.

\subsubsection{\texttt{+ setAcceptPrivacyPolicy(value : boolean)}}%$$$M
\paragraph*{Beschreibung}
Speichert, ob der Benutzer den Datenschutz akzeptiert hat.
\paragraph*{Parameter}
\begin{itemize}
    \item value : boolean Wert, ob der Benutzer den Datenschutz akzeptiert hat.
\end{itemize}
\paragraph*{Rückgabewert}
Void.

\subsubsection{\texttt{+ isPrivacyAccepted() : Livedata<boolean>}}%$$$M
\paragraph*{Beschreibung}
Gibt den Wert zurück, ob der Benutzer den Datenschutz akzeptiert hat.
\paragraph*{Parameter}
keine.
\paragraph*{Rückgabewert}
Livedata<boolean>, ob der Benutzer den Datenschutz akzeptiert hat.

\subsubsection{\texttt{+ signIn(provider : String) : boolean}}%$$$M
\paragraph*{Kurzbeschreibung}
Leitet den Benutzer zur Eingabe seiner Daten weiter.
\paragraph*{Beschreibung}
Ausgelöst, wenn der Benutzer einen Anmeldedienst auswählt. 
Der Aufruf wird an das Model weitergegeben.
\paragraph*{Parameter}
\begin{itemize}
    \item provider : String ausgewählter Anmeldedienst.
\end{itemize}
\paragraph*{Rückgabewert}
boolean, ob die Anmeldung erfolgreich war.

%----------------------------------------------------------
\subsubsection{SignInController}
\paragraph*{Typ}
Implementiert ISignInController.
\paragraph*{Beschreibung}
Stellt Daten für das Anzeigen der Anmeldung bereit und verarbeitet die Interaktionen.\\
Besitzt für die Datenbeschaffung und die Ausführung der Anmeldung einen SignInModel.
