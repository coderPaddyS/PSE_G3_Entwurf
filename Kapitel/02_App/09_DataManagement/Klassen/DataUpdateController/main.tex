\subsubsection{IDataUpdateController}\label{App_DocumentViewer_IDataupdateController}
\paragraph*{Typ}
\texttt{Interface}.
\paragraph*{Beschreibung}
Ein DataUpdateController aktualisiert die lokale Datenbank. Das Aktualisieren geschieht im 
Hintergrund. Daher wird er parallel zur eigentlichen Programmausführung aufgerufen.\\
Ein DataUpdateController besitzt eine Liste von \hyperref[App_DocumentViewer_DataUpdater]{DataUpdatern}.
Es gibt eine Warteschlange von \hyperref[App_DocumentViewer_DataUpdater]{DataUpdatern}, in der die Priorität der Schlüssel ist (je kleiner 
die Priorität, desto früher kommt ein \hyperref[App_DocumentViewer_DataUpdater]{DataUpdater} dran). Initial ist die Warteschlange leer.\\


\subsubsection*{\texttt{+ resumeUpdate() : boolean}}\label{App_DocumentViewer_IDataupdateController_resumeUpdate}%$$$M
\paragraph*{Kurzbeschreibung}
Führt die Aktualiserung fort.
\paragraph*{Beschreibung}
Diese Methode kann dazu verwendet werden, eine fehlgeschlagene Aktualisierung fortzusetzen.\\
Diese Methode führt alle \hyperref[App_DocumentViewer_DataUpdater]{DataUpdater} in der Warteschlange aus.\\
War das Aktualsieren eines \hyperref[App_DocumentViewer_DataUpdater]{DataUpdaters} erfolgreich, wird er aus der Warteschlange entfernt.\\
Schlägt ein DataUpdater fehl, werden nur noch alle \hyperref[App_DocumentViewer_DataUpdater]{DataUpdater} mit gleicher Priorität 
ausgeführt. Danach wird false zurückgegeben.\\
Ist die Warteschlange leer, wird true zurückgegeben.\\
\paragraph*{Parameter}
Keine.
\paragraph*{Rückgabewert}
\texttt{boolean}, ob die Aktualisierung erfolgreich war.

\subsubsection*{\texttt{+ updateDatabase() : boolean}}\label{App_DocumentViewer_IDataupdateController_updateDatabase}%$$$M
\paragraph*{Kurzbeschreibung}
Aktualisiert die lokale Datenbank komplett.
\paragraph*{Beschreibung}
Diese Methode aktualisiert die lokale Datenbank komplett. 
Dazu wird die Warteschlange geleert und alle DataUpdater hinzugefügt.\\
Danach werden alle \hyperref[App_DocumentViewer_DataUpdater]{DataUpdater} in der Warteschlange ausgeführt.\\
War das Aktualisieren eines \hyperref[App_DocumentViewer_DataUpdater]{DataUpdaters} erfolgreich, wird er aus der Warteschlange entfernt.\\
Schlägt ein \hyperref[App_DocumentViewer_DataUpdater]{DataUpdater} fehl, werden nur noch alle \hyperref[App_DocumentViewer_DataUpdater]{DataUpdater} mit gleicher Priorität 
ausgeführt. Danach wird false zurückgegeben.\\
Konnten alle \hyperref[App_DocumentViewer_DataUpdater]{DataUpdater} ausgeführt werden, wird true zurückgegeben.
\paragraph*{Parameter}
Keine.
\paragraph*{Rückgabewert}
\texttt{boolean}, ob die Aktualisierung erfolgreich war.

%---------------------------------------------------------
\subsubsection{DataUpdateController}\label{App_DocumentViewer_DataupdateController}
\paragraph*{Typ}
Implementiert \texttt{IDataUpdateController}.
\paragraph*{Beschreibung}
Diese Klasse implementiert einen \href{App_DocumentViewer_IDataupdateController}{IDataUpdateController}.\\
Die Liste an zu aktualisierenden \hyperref[App_DocumentViewer_DataUpdater]{DataUpdatern} enthält einen GlobalAliasUpdater.