\subsubsection{IDataUpdateController}\label{App_DocumentViewer_IDataupdateController}
\paragraph*{Typ}
\texttt{Interface}.
\paragraph*{Beschreibung}
Ein DataUpdateController aktualisiert die lokale Datenbank. Das Aktualisieren geschieht im 
Hintergrund. Daher wird er parallel zur eigentlichen Programmausführung aufgerufen.\\
Ein DataUpdateController besitzt eine Liste von DataUpdatern.
Es gibt eine Warteschlange von DataUpdatern, in der die Priorität der Schlüssel ist (je kleiner 
die Priorität, desto früher kommt ein DataUpdater dran). Initial ist die Warteschlange leer.\\


\subsubsection*{\texttt{+ resumeUpdate() : boolean}}\label{App_DocumentViewer_IDataupdateController_resumeUpdate}%$$$M
\paragraph*{Kurzbeschreibung}
Führt die Aktualiserung fort.
\paragraph*{Beschreibung}
Diese Methode kann dazu verwendet werden, eine fehlgeschlagene Aktualisierung fortzusetzen.\\
Diese Methode führt alle DataUpdater in der Warteschlange aus.\\
War das Aktualsieren eines DataUpdaters erfolgreich, wird er aus der Warteschlange entfernt.\\
Schlägt ein DataUpdater fehl, werden nur noch alle DataUpdater mit gleicher Priorität 
ausgeführt. Danach wird false zurückgegeben.\\
Ist die Warteschlange leer, wird true zurückgegeben.\\
\paragraph*{Parameter}
Keine.
\paragraph*{Rückgabewert}
\texttt{boolean}, ob die Aktualisierung erfolgreich war.

\subsubsection*{\texttt{+ updateDatabase() : boolean}}\label{App_DocumentViewer_IDataupdateController_updateDatabase}%$$$M
\paragraph*{Kurzbeschreibung}
Aktualisiert die lokale Datenbank komplett.
\paragraph*{Beschreibung}
Diese Methode aktualisiert die lokale Datenbank komplett. 
Dazu wird die Warteschlange geleert und alle DataUpdater hinzugefügt.\\
Danach werden alle DataUpdater in der Warteschlange ausgeführt.\\
War das Aktualisieren eines DataUpdaters erfolgreich, wird er aus der Warteschlange entfernt.\\
Schlägt ein DataUpdater fehl, werden nur noch alle DataUpdater mit gleicher Priorität 
ausgeführt. Danach wird false zurückgegeben.\\
Konnten alle DataUpdater ausgeführt werden, wird true zurückgegeben.
\paragraph*{Parameter}
Keine.
\paragraph*{Rückgabewert}
\texttt{boolean}, ob die Aktualisierung erfolgreich war.

%---------------------------------------------------------
\subsubsection{DataUpdateController}\label{App_DocumentViewer_DataupdateController}
\paragraph*{Typ}
Implementiert \texttt{IDataUpdateController}.
\paragraph*{Beschreibung}
Diese Klasse implementiert einen IDataUpdateController.\\
Die Liste an zu aktualisierenden DataUpdatern enthält einen GlobalAliasUpdater.