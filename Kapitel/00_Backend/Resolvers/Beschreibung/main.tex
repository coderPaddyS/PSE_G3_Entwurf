\subsection*{Paketbeschreibung}%$$$M
\paragraph*{Kurzbeschreibung}
Beinhaltet die Schnittstellen, die von außen durch Http-Requests über die GraphQL-Schnittstelle aufgerufen werden können.
\paragraph*{Beschreibung}
Der Aufruf der Methoden der Klassen dieses Paketes geschieht durch die in den Resourcen definierte GraphQL-Schnittstelle.
Wird GraphQL nicht genutzt, so muss auf andere Weise sichergestellt werden, dass die jeweiligen Http-Request zur Ausführung der richtigen Methoden
führen und die richtige Rückgabe erzeugt wird.
\paragraph*{Enthaltene Pakete}
\begin{itemize}
    \item MutationResolver
    		\paragraph*{Beschreibung}
        Bietet Schnittstellen zur Manipulation der gespeicherten Daten.
    \item QueryResolver
    		\paragraph*{Beschreibung}
    		Bietet Schnittstelle zum Abfragen gespeicherter Daten.
\end{itemize}

\begin{figure}
  \centering
  \includegraphics[width=\textwidth]{\currfiledir Resolvers}
\end{figure}