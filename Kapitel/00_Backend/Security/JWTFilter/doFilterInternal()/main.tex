\paragraph{\texttt{doFilterInternal(HttpServletRequest request, HttpServletResponse response, FilterChain filterChain)}}%$$$M
\paragraph*{Kurzbeschreibung}
Führt den Filter aus. Hierbei wird ggf. anhand des Access-Tokens Authorisierung bzw. Authentifizierung vorgenommen.
\paragraph*{Beschreibung}
Zunächst wird ggf. der Token dem Header des Http-Requests entnommen. Dieser wird anschließend mithilfe des in
\dq SecurityConfig \dq definierten Verifiers validiert. Ist er gültig, so wird zunächst der Nutzeridentifikator dem Token entnommen und mittels der \dq loadByToken \dq Methode 
des UserService der entsprechende User geladen. Ist dieser noch nicht vorhanden, so muss dieser mit der \dq addUser \dq Methode
zuerst hinzugefügt und dann erneut geladen werden. Anschließend werden die Authorisierungen des Users einem \dq 
JWTPreAuthenticationToken \dq übergeben. Anschließend wird mit der \dq getContext \dq Methode des SecurityContextHolders der 
Security-Context geladen und diesem mittels der \dq setAuthentication() \dq Methode der erstellte JWTPreAuthenticationToken 
übergeben. 

\paragraph*{Parameter}
\begin{itemize}
    \item HttpServletRequest
    		\paragraph*{Beschreibung}
            Von einem Frontend abgesandter, eintreffender HttpRequest, der ggf. einen Access-Token im Header trägt.
    \item HttpServletResponse
        	\paragraph*{Beschreibung}
            Genutzt um den HttpRequest nach Beenden des Filters weiterzuleiten.
    \item FilterChain
    		\paragraph*{Beschreibung}
            Klasse, die dafür sorgt, dass alle vorangeschaltenen Filter in einer festen Reihenfolge ausgeführt werden. Da es in diesem Entwurf nur einen Filter gibt,
            sorgt dieser dementsprechend nur für die Ausführung des einen Filters und der Request wird nach dessen Ausführung direkt (an die GraphQL-Schnittstelle) weitergeleitet.

\end{itemize}
