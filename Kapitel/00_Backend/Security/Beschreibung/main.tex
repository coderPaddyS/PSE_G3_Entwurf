\subsection*{Paketbeschreibung}%$$$M
\paragraph*{Kurzbeschreibung}
Beinhaltet alle Klassen und Konfigurationen, die zur Authentifizierung und Authorisierung des Nutzers nötig sind.
\paragraph*{Beschreibung}
Bestimmte Methoden auf Resolverebene sollen nur angemeldeten Nutzern oder teilweise auch nur Nutzern, die bestimmte Authorisierungen
innehaben, zugänglich sein. Um dies mittels Spring-Security zu realisieren wird in \dq WebSecurityConfig \dq spezifiziert, dass mittels eines vorangeschaltenen Filters vor jedem
Methodenaufruf der Header des zu bearbeitenden Http-Requests auf ein Access-Token überprüft werden soll. Ist ein valider Token enthalten, 
wird diesem die Nutzeridentifikation entnommen. Mit diesem eindeutigen KKennzeichner wird die entsprechende User-Entitys
wenn noch nicht vorhanden erstellt und geladen. Dieser werden dann  die Nutzerauthoritäten entnommen und in ein JWTPreAuthenticationToken gewrapped in den Security-Kontext abgelegt. Dadurch kann vor der Methodenausführung
im Resolver überprüft werden, ob der Security-Context die benötigte Authorität enthält.

\paragraph*{Enthaltene Klasse}
\begin{itemize}
	\item WebSecurityConfig
	\paragraph*{Beschreibung}
			Klasse, in der die Standardkonfigurationen der Zugriffsberechtigungen auf die Methoden 
			der Resolverebene festgelegt werden (diesen gelten, wenn den Methoden nicht durch Annotationen o.ä. andere
			Authoritäten zugeordnet werden). Außerdem wird hier für das Voranschalten des Filters zur Authorisierung und Authentifizierung gesorgt.
    \item JWTFilter
    		\paragraph*{Beschreibung}
    		Überprüft HttpRequest auf ein gültiges Token und fügt gegebenfalls ein JWTPreAuthenticationToken mit den entsprechenden Authoritäten in den Security-Context ein.
	\item SecurityConfig
			\paragraph*{Beschreibung}
			Klasse, in der der Algorithmus und mittels diesem auch der Verifizier für die Signatur gegebener Access-Token definiert werden.		
	\item JWTPreAuthenticationToken
    		\paragraph*{Beschreibung}
			Wrapper-Klasse für gegebene Authoritäten.
\end{itemize}
