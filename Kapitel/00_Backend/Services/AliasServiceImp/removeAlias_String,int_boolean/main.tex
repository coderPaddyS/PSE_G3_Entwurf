\paragraph{\texttt{removeAlias(alias : String, mapID : int) : boolean}}%$$$M
\paragraph*{Kurzbeschreibung}
Erteilt die Anweisung an Klasse \texttt{AliasRepository}, einen globalen Alias aus der Datenbank zu entfernen.
\paragraph*{Beschreibung}
Die Methode implementiert die gleichnamige Methode der Schnittstelle \texttt{AliasService}.
Die Methode ist serverintern und wird somit nicht explizit vom Administrator aufgerufen.
Stattdessen wird die Methode von Methoden aus dem Paket \texttt{Resolvers} aufgerufen.
Die Methode ermöglicht, dass ein Administrator einen globalen Alias aus der Datenbank aus allen globalen Aliassen entfernen kann.
Dies hat zur Folge, dass das entsprechende Kartenobjekt nicht mehr durch diesen, nun gelöschten Alias, gefunden werden kann.
Hierfür wird der gewünschte globale Alias als Zeichenkette als auch die ID des dazugehörigen Kartenobjekts übergeben.
\paragraph*{Parameter}
\begin{itemize}
    \item alias : String
    		\paragraph*{Beschreibung}
    		Der globale Alias als Zeichenkette, den man entfernen möchte.
    \item mapID : int
    		\paragraph*{Beschreibung}
    		Die ID des Kartenobjekts, für das der globale Alias als Bezeichner dient.
\end{itemize}
\paragraph*{Rückgabewert}
Wahr, wenn der gewünschte globale Alias erfolgreich entfernt werden konnte, ansonsten falsch.
