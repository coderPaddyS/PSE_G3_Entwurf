\paragraph*{\texttt{+doesAliasExist(mapID : int, alias : String) : Boolean}}%$$$M
\paragraph*{Kurzbeschreibung}
Erteilt die Anweisung an Klasse \texttt{AliasRepository}, die Existenz eines Aliasses zu überprüfen, und gibt diesen Wahrheitswert zurück.
\paragraph*{Beschreibung}
Die Methode implementiert die gleichnamige Methode der Schnittstelle \texttt{AliasService}.
Die Methode ist serverintern und wird somit nicht explizit vom Administrator aufgerufen.
Stattdessen wird die Methode von Methoden aus dem Paket \texttt{Resolvers} aufgerufen.
Die Methode ermöglicht, dass ein Administrator anfragen kann, ob ein Alias existiert.
Hierfür wird die ID des dazugehörigen Kartenobjekts und der gewünschte Alias als Zeichenkette übergeben.
\paragraph*{Parameter}
\begin{itemize}
    \item mapID : int
    		\paragraph*{Beschreibung}
    		Die ID des Kartenobjekts, für das der Alias als Bezeichner dient.
    	\item alias : String
    		\paragraph*{Beschreibung}
    		Der Alias als Zeichenkette, dessen Existenz man überprüfen will.
\end{itemize}
\paragraph*{Rückgabewert}
Wahr, wenn der übergebene Alias existiert, ansonsten falsch.
