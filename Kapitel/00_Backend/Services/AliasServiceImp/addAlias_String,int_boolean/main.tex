\paragraph{\texttt{addAlias(alias : String, mapID : int) : boolean}}%$$$M
\paragraph*{Kurzbeschreibung}
Erteilt die Anweisung an Klasse \texttt{AliasRepository}, einen Alias zur Datenbank der globalen Aliasse hinzuzufügen.
\paragraph*{Beschreibung}
Die Methode implementiert die gleichnamige Methode der Schnittstelle \texttt{AliasService}.
Die Methode ist serverintern und wird somit nicht explizit vom Administrator aufgerufen.
Stattdessen wird die Methode von Methoden aus dem Paket \texttt{Resolvers} aufgerufen.
Die Methode ermöglicht, dass ein Administrator einen Alias zur Datenbank der globalen Aliassen hinzufügen kann.
Dies hat zur Folge, dass das entsprechende Kartenobjekt nun zusätzlich durch diesen neuen hinzugefügten Alias gefunden werden kann.
Hierfür wird der gewünschte Alias als Zeichenkette als auch die ID des dazugehörigen Kartenobjekts übergeben.
\paragraph*{Parameter}
\begin{itemize}
    \item alias : String
    		\paragraph*{Beschreibung}
    		Der Alias als Zeichenkette, den man hinzufügen möchte.
    \item mapID : int
    		\paragraph*{Beschreibung}
    		Die ID des Kartenobjekts, für das der globale Alias als Bezeichner dient.
\end{itemize}
\paragraph*{Rückgabewert}
Wahr, wenn der gewünschte globale Alias erfolgreich hinzugefügt werden konnte, ansonsten falsch.
