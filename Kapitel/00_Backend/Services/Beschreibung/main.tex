\subsection*{Paketbeschreibung}%$$$M
\paragraph*{Kurzbeschreibung}
Schnittstellen, welche die Datenhaltung abstrahieren und Zugriff auf die zugrundeliegende Daten ermöglichen, ohne deren Struktur offenszulegen.
Hierfür werden Interfaces genutzt, die von den Resolvern referenziert werden, ebenfalls um in den oberen Schichten möglichst 
wenig von Implementierungsdetails abhängig zu sein. Für die Datenhaltung eignet sich ein JpaRepository, dass auf 
eine SQL Datenbank zugreift, sodass wir diese Variante empfehlen. Dementsprechend benötigt jede Service-Implementierungsklasse die jeweilige 
zugrundeliegende Repositoryklasse um ihre Methoden umzusetzen. 
\paragraph*{Enthaltene KLassen}
\begin{itemize}
    \item AliasService <<interface>>
    \item AliasServiceImp
    \item DeletedAliasService <<interface>>
    \item DeletedAliasServiceImp
    \item UserService <<interface>>
    \item UserServiceImp
    \item AliasSuggestionService <<interface>>
    \item AliasSuggestionServiceImp
    \item BlacklistService <<interface>>
    \item BlacklistServiceImp
    \item MapIDService <<interface>>
    \item MapIDServiceImp
\end{itemize}

