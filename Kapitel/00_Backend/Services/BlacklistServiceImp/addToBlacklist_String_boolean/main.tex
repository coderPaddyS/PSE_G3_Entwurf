\paragraph{\texttt{addToBlacklist(toBlacklist : String) : boolean}}%$$$M
\paragraph*{Kurzbeschreibung}
Erteilt die Anweisung an Klasse \texttt{BlacklistRepository}, eine Zeichenkette in die Blacklist aufzunehmen.
\paragraph*{Beschreibung}
Die Methode implementiert die gleichnamige Methode der Schnittstelle \texttt{BlacklistService}.
Die Methode ist serverintern und wird somit nicht explizit vom Administrator aufgerufen.
Stattdessen wird die Methode von Methoden aus dem Paket \texttt{Resolvers} aufgerufen.
Die Methode ermöglicht, dass ein Administrator eine Zeichenkette in die Blacklist aufnehmen kann.
Dies hat zur Folge, dass ein Alias-Vorschlag, der genau aus dieser Zeichenkette zusammengesetzt ist, nicht angenommen werden kann und herausgefiltert wird.
Hierfür wird die in die Blacklist aufzunehmende Zeichenkette übergeben.
\paragraph*{Parameter}
\begin{itemize}
    \item toBlacklist : String
    		\paragraph*{Beschreibung}
    		Die in die Blacklist aufzunehmende Zeichenkette.
\end{itemize}
\paragraph*{Rückgabewert}
Wahr, wenn die übergebene Zeichenkette in die Blacklist erfolgreich aufgenommen werden konnte oder die Zeichenkette bereits in der Blacklist hinterlegt war, ansonsten falsch.
