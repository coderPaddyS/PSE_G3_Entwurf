\subsubsection{\texttt{loadUserByToken(accessToken : String) : User}}%$$$M
\paragraph*{Kurzbeschreibung}
Ladet den gewünschten Benutzer durch einen Access-Token und gibt ihm anschließend zurück.
\paragraph*{Beschreibung}
Die Methode implementiert die gleichnamige Methode der Schnittstelle \texttt{UserService}.
Die Methode ist serverintern und wird somit nicht explizit vom Administrator aufgerufen.
Stattdessen wird die Methode von Methoden aus dem Paket \texttt{Resolvers} aufgerufen.
Die Methode ermöglicht, dass ein Administrator durch Angabe des Access-Tokens den gewünschten Benutzer erhalten kann. 
Ist der Access-Token ungültig, d.h. nicht von einem Benutzer, wird \texttt{null} zurückgegeben.
\paragraph*{Parameter}
\begin{itemize}
    \item accessToken : String
    		\paragraph*{Beschreibung}
    		Der Access-Token des gewünschten Benutzers, der zurückgegeben werden soll.
\end{itemize}
\paragraph*{Rückgabewert}
Der Benutzer, zu dem der übergebene Access-Token gehört, sofern der übergebene Access-Token gültig ist.
Sollte der übergebene Access-Token ungültig sein, wird \texttt{null} zurückgegeben.
