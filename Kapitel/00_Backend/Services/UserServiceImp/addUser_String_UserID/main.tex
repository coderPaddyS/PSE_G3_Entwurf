\paragraph{\texttt{addUser(accessToken : String) : UserID}}%$$$M
\paragraph*{Kurzbeschreibung}
Nimmt einen Benutzer, der sich authentifiziert hat und noch nicht in die Datenbank aufgenommen wurde, in der Datenbank auf.
\paragraph*{Beschreibung}
Die Methode implementiert die gleichnamige Methode der Schnittstelle \texttt{UserService}.
Die Methode ist serverintern und wird somit nicht explizit vom Administrator aufgerufen.
Stattdessen wird die Methode von Methoden aus dem Paket \texttt{Resolvers} aufgerufen.
Die Methode ermöglicht, dass ein Administrator einen Benutzer zur Datenbank aus allen Benutzern hinzufügen kann. Hierfür muss der Benutzer sich authentifiziert haben. 
Zudem wird der Benutzer nur dann in die Datenbank aufgenommen, wenn er noch nicht in der Datenbank hinterlegt ist.
Hierfür wird der Access-Token des gewünschten Benutzers übergeben.
\paragraph*{Parameter}
\begin{itemize}
    \item accessToken : String
    		\paragraph*{Beschreibung}
    		Der Access-Token des gewünschten Benutzers, der sich bereits authentifiziert hat und den man in die Datenbank aufnehmen möchte.
\end{itemize}
\paragraph*{Rückgabewert}
Die ID des in die Datenbank aufgenommenen Benutzers.
