\paragraph*{\texttt{+getAlias(mapID : int, user : String) : Iterable<alias : String>}}%$$$M
\paragraph*{Kurzbeschreibung}
Definiert das Erteilen der Anweisung an Klasse \texttt{AliasRepository}, eine Liste der Aliasse des gewünschten Kartenobjekts, die ein bestimmter Benutzer ursprünglich vorgeschlagen hatte, zurückzugeben, und gibt diese wiederum zurück.
\paragraph*{Beschreibung}
Die Methode ist serverintern und wird somit nicht explizit vom Administrator aufgerufen.
Stattdessen wird die Methode von Methoden aus dem Paket \texttt{Resolvers} aufgerufen.
Die Methode ermöglicht, dass ein Administrator eine Liste aus allen Aliassen eines Kartenobjekts anfragen kann, die ein bestimmter Benutzer ursprünglich vorgeschlagen hatte.
D.h., durch die Methode erhält man alle Alias-Vorschläge des Benutzers, die bereits von einem Administrator angenommen und somit zu globalen Aliasse verwandelt wurden.
Hierfür wird die ID des gewünschten Kartenobjekts und der Identifikator des gewünschten Benutzers übergeben.
\paragraph*{Parameter}
\begin{itemize}
    \item mapID : int
    		\paragraph*{Beschreibung}
    		Die ID des Kartenobjekts, dessen Aliasse man zurückgegeben haben möchte.
    	\item user : String
    		\paragraph*{Beschreibung}
    		Der Identifikator des Benutzers, dessen Aliasse man zurückgegeben haben möchte.
\end{itemize}
\paragraph*{Rückgabewert}
Die Liste mit den Aliassen des gewünschten Kartenobjekts, die der Benutzer ursprünglich vorgeschlagen hatte.
