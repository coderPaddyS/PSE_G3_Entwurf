\paragraph{\texttt{removeFromBlacklist(blacklistedToRem : String) : boolean}}%$$$M
\paragraph*{Kurzbeschreibung}
Definiert das Erteilen der Anweisung an Klasse \texttt{BlacklistRepository}, eine Zeichenkette aus der Blacklist zu entfernen.
\paragraph*{Beschreibung}
Die Methode ist serverintern und wird somit nicht explizit vom Administrator aufgerufen.
Stattdessen wird die Methode von Methoden aus dem Paket \texttt{Resolvers} aufgerufen.
Die Methode ermöglicht, dass ein Administrator eine Zeichenkette aus der Blacklist entfernen kann.
Dies hat zur Folge, dass ein Alias-Vorschlag, der genau aus dieser Zeichenkette zusammengesetzt ist, nicht mehr herausgefiltert wird und somit die Möglichkeit hat, angenommen zu werden.
Hierfür wird die aus der Blacklist zu entfernende Zeichenkette übergeben.
\paragraph*{Parameter}
\begin{itemize}
    \item blacklistedToRem : String
    		\paragraph*{Beschreibung}
    		Die aus der Blacklist zu entfernende Zeichenkette.
\end{itemize}
\paragraph*{Rückgabewert}
Wahr, wenn die übergebene Zeichenkette aus der Blacklist erfolgreich entfernt werden konnte oder die Zeichenkette bereits nicht in der Blacklist hinterlegt war, ansonsten falsch.
