\subsection*{Paketbeschreibung}%$$$M
\paragraph*{Kurzbeschreibung}
    Enthält die Klassen der zu speichernden Entities.
\paragraph*{Anmerkung}  
    Die hier beschriebenenen Getter- und Settermethoden sind nur die, welche für eine anwendungsinterne Datenhaltung 
    notwendig sind. Für eine Datenhaltung in einer Datenbank mittels Spring Data JPA und Hibernate, die wir bevorzugen, sollten 
    standardmäßige Getter und Setter für alle Atribute der Entities generiert werden.
\paragraph*{Enthaltene Klassen}
\begin{itemize}
    \item Alias
    		\paragraph*{Beschreibung}
            Als bestehender Alias für ein Kartenobjekt in der entsprechenden Datenhaltung gespeichert und von 
            dem AliasRepository verwaltet oder als gelöschter Alias von DeletedAliasRepository verwaltet.
    \item AliasSuggestion
    		\paragraph*{Beschreibung}
          Bestehender Alias-Vorschlag für ein Kartenobjekt, der bewertet und akzeptiert werden kann.
    \item BlacklistEntry 
            \paragraph{Beschreibung}
          Bestehender Eintrag in der Blacklist.
    \item MapID
            \paragraph{Beschreibung} 
          Bildet eine bestimmte MapId vom Typ int auf das entsprechende Kartenobjekt ab.
\end{itemize}

\begin{figure}
  \centering
  \includegraphics[width=\textwidth]{\currfiledir Models}
  \caption{UML-Paketdiagramm}
\end{figure}