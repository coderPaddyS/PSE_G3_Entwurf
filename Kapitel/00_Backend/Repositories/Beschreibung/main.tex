\subsection*{Klassenbeschreibung}%$$$M
\paragraph*{Kurzbeschreibung}
Wie erwähnt empfehlen wir die Datenhaltung mittels SQL Datenbanken, und den Zugriff auf diese mittels eines Repsitories,
dass von der Libraryklasse \dq JpaRepository \dq erbt. Hierfür wird eine seperate Klasse mit den jeweiligen Konfigurationen benötigt,
die dafür sorgt, dass die richtigen Einträge der richtigen Datenbank und dem passenden Repsoitory zugeordnet werden können (siehe Paket \dq Configurations \dq).

\paragraph*{Enthaltene Klassen}
\begin{itemize}
    \item AliasRepsoitory
    \item AliasSuggestionRepository
    \item BlacklistRepository
    \item MapIDRepository
    \item UserRepository
    \item DeletedAliasRepository
\end{itemize}
