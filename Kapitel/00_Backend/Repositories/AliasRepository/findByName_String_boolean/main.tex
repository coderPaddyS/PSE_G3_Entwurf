\paragraph{\texttt{findByName(alias : String) : int}}%$$$M
\paragraph*{Kurzbeschreibung}
Gibt die ID eines in der Datenbank hinterlegten Aliasses zurück.
\paragraph*{Beschreibung}
Die Methode ist serverintern und wird somit nicht explizit vom Administrator aufgerufen.
Stattdessen wird die Methode von Methoden aus dem Paket \texttt{Services} aufgerufen.
Die Methode ermöglicht, dass die ID eines bestimmten Aliasses durch Angabe der entsprechenden Zeichenkette angefragt und zurückbekommen werden kann.
Hierfür wird der Alias als Zeichenkette übergeben.
\paragraph*{Parameter}
\begin{itemize}
    \item alias : String
    		\paragraph*{Beschreibung}
    		Der als Zeichenkette angegebene Alias.
\end{itemize}
\paragraph*{Rückgabewert}
Die ID des als Zeichenkette übergebenen Aliasses, sofern die übergebene Zeichenkette gültig ist.
Im Fall, dass die übergebene Zeichenkette ungültig ist, wird \texttt{null} zurückgegeben.
