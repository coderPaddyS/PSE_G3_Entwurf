\paragraph{\texttt{findByName(user : String) : int}}%$$$M
\paragraph*{Kurzbeschreibung}
Gibt die ID eines bestimmten in der Datenbank hinterlegten Benutzers zurück.
\paragraph*{Beschreibung}
Die Methode ist serverintern und wird somit nicht explizit vom Administrator aufgerufen.
Stattdessen wird die Methode von Methoden aus dem Paket \texttt{Services} aufgerufen.
Die Methode ermöglicht, dass die ID eines bestimmten Benutzers durch Angabe der entsprechenden Zeichenkette, d.h. des vom Access-Token abgeleitete Benutzeridentifikators, angefragt und zurückgegeben werden kann.
Hierfür wird der vom Access-Token abgeleitete Benutzeridentifikator als Zeichenkette übergeben.
\paragraph*{Parameter}
\begin{itemize}
    \item user : String
    		\paragraph*{Beschreibung}
    		Der vom Access-Token abgeleitete Benutzeridentifikator.
\end{itemize}
\paragraph*{Rückgabewert}
Die ID des als Zeichenkette übergebenen Benutzers, sofern die übergebene Zeichenkette gültig ist.
Im Fall, dass die übergebene Zeichenkette ungültig ist, wird \texttt{null} zurückgegeben.
