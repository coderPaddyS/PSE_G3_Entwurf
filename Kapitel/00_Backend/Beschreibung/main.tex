\subsection{Beschreibung}%$$$M

\paragraph*{Backend}
Das \dq Backend \dq stellt die Serverseite unserer Applikation dar. Sie ist verantwortlich für die Datenhaltung und muss die benötigten Schnittstellen für Applikation und Admin-Panel zur Verfügung stellen.
 Hierbei muss jedoch zwischen öffentlich zugänglichen, nur für angemeldete Nutzer verfügbare und Admin beschränkten Schnittstellen differenziert werden.
\paragraph*{Architekturbeschreibung}
Das Server-Backend wird in einer 3-stufigen, durchlässigen Schichtenarchitektur strukturiert. Nachdem die Authorisierung und Authentifizierung
mittels eines vorangeschaltenen Filters erfolgt ist, wird der Http-Request an eine in den Ressourcen definierte GraphQL-Schnittstelle weitergeleitet. Diese Schnittstelle
entspricht dem Entwurfsmuster \dq Fassade \dq. Diese einheitliche Schnittstelle bietet den Zugriff auf eine Menge von Schnittstellen der obersten Schicht - den Resolvern.
Hier wird vor Methodenaufruf auf die benötigten Berechtigungen überprüft. Sind diese erfüllt, werden entsprechende Schnitstellen der darunterliegenden Schicht - den Sevices - 
für die Umsetzung genutzt. Die Services bieten Zugang zu den hinterlegten Daten. Für die Referenzierung dieser auf Resolverebene werden Interfaces genutzt, um die Datenhaltung 
von der Anwendungslogik zu trennen. Es ist anzumerken, dass hier nicht von einer undurchlässigen Schichtenarchitektur gesprochen werden kann, da der in gewisser Weise
über der obersten Schicht liegende Filter den Service für die Datenhaltung der Nutzer benötigt, um dessen Berechtigungen bestimmen zu können.
Um eine Datenhaltung mittels einer Datenbank zu realisieren, was nicht unbedingt notwendig, aber empfohlen wird, ist eine weitere Schicht von Nöten: die Repositories.
Diese setzen den konkreten Umgang mit der zugrundeliegenden Datenbank um und ermöglichen Lese- und Schreibzugriffe, sodass benötigte Informationen abgefragt und bestehende manipuliert 
werden können.
 
