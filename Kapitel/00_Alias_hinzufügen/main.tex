\newpage
\section{Alias Hinzufügen}

\subsection{lokales Hinzufügen}

Für das Hinzufügen von Alias und Abfragen in einer lokalen Datenbank eignet sich SQL. Hierbei werden
ein SQL-Helper und eine Datenbank benötigt. In der Datenbank müssen der Name und das zugehörige Gebäude und ggf. 
der Raum dazu hinterlegt werden. Es müssen funktionalitäten zum Erstellen, Finden únd Löschen von Aliassen zur Verfügung stehen.
Es sollte zwischen Raum-/Gebäudenummer und Aliassen unterschieden werden, da offizielle nicht gelöscht werden können sollen.

\subsection{globales Hinzufügen}

Hierfür wird eine remote SQL-Datenbank erfordert. Der Zugriff auf diese sollte über einen Webservice (PHP) erfolgen.
Die App auf den Geräten der Nutzer sollte mit der remote DB synchronisiert werden. Hierfür würde sich das SyncAdapter framework
von Android eignen. Sucht der Nutzer in der Applikationen nach etwas, sollte zuerst die lokale Datenbank dursucht werden und 
erst im Anschluss, wenn das Ziel noch nicht gefunden wurde, die remote Datenbank.