\chapter{Grundbausteine}

\section{Alias}
\paragraph*{Beschreibung}
Ein alternativer Bezeichner für ein Kartenobjekt (MapObject), wie beispielsweise ein Raum (Room) oder Gebäude (Building).
Benutzer können mit diesem nach dem zugehörigen Kartenobjekt suchen alternativ zum offizielen Namen (OfficialName).
Aliasse können lokal sein (Benutzer hat den Alias selbst für sich hinzugefügt) oder global (Alias wurde von einem Benutzer vorgeschlagen und von einem Admin für alle Benutzer hinzugefügt).
Die Aliasse werden lokal und auf dem Server gespeichert.

\subsection{isLocal}%$$$M
\paragraph*{Kurzbeschreibung}
Diese Methode gibt zurück, ob der Alias lokal ist.
\paragraph*{Beschreibung}
Diese Methode gibt zurück, ob der Alias lokal ist (wahr) oder nicht (falsch).
\paragraph*{Parameter}
\begin{itemize}
    \item keine
\end{itemize}
\paragraph*{Rückgabewert}
bool ob der Alias lokal ist

\subsection{getIdentifier}%$$$M
\paragraph*{Kurzbeschreibung}
Diese Methode gibt den Bezeichner des Alias zurück.
\paragraph*{Beschreibung}
Diese Methode gibt den Bezeichner des Alias zurück.
\paragraph*{Parameter}
\begin{itemize}
    \item keine
\end{itemize}
\paragraph*{Rückgabewert}
String Bezeichner des Alias

\subsection{getMapId}%$$$M
\paragraph*{Kurzbeschreibung}
Diese Methode gibt die KartenID des Alias zurück.
\paragraph*{Beschreibung}
Diese Methode gibt die KartenID des Alias zurück.
\paragraph*{Parameter}
\begin{itemize}
    \item keine
\end{itemize}
\paragraph*{Rückgabewert}
MapId KartenID des Alias


\section{Person}
\paragraph*{Beschreibung}
Eine Person, die ein Büro in einem Gebäude hat. Der Name der Person dient als alternativer Identifikator für einen Raum.
\paragraph*{Anmerkung}
Die Personen werden lokal und auf dem Server gespeichert.

\subsection{getName}%$$$M
\paragraph*{Kurzbeschreibung}
Diese Methode liefert den vollständigen Namen der Person.
\paragraph*{Beschreibung}
Diese Methode liefert den vollständigen Namen der Person einschließlich des Titels in der Reihenfolge "Titel Vorname Nachname".
\paragraph*{Parameter}
\begin{itemize}
    \item keine
\end{itemize}
\paragraph*{Rückgabewert}
String Name der Person

\subsection{getMapId}
\paragraph*{Kurzbeschreibung}
Diese Methode gibt die KartenID der Person zurück.
\paragraph*{Beschreibung}
Diese Methode gibt die KartenID der Person zurück.
\paragraph*{Parameter}
\begin{itemize}
    \item keine
\end{itemize}
\paragraph*{Rückgabewert}
MapId KartenID


\section{Building}
\paragraph*{Beschreibung}
Ein Gebäude mit Räumen, nach denen in der App gesucht werden kann.
\paragraph*{Anmerkung}
Die Gebäude werden lokal und auf dem Server gespeichert.


\section{Room}
\paragraph*{Kurzbeschreibung}
Ein Raum in einem Gebäude.
\paragraph*{Beschreibung}
Ein Raum in einem Gebäude, nach dem in der App gesucht werden kann.
\paragraph*{Anmerkung}
Die Räume werden lokal und auf dem Server gespeichert.


\section{MapObject}
\paragraph*{Kurzbeschreibung}
Ein Kartenobjekt.
\paragraph*{Beschreibung}
Ein Kartenobjekt, nach dem in der App gesucht werden kann.
\paragraph*{Anmerkung}
Die Kartenobjekte werden lokal und auf dem Server gespeichert.

\subsection{getAdditionalData}%$$$M
Diese Methode gibt die zusätzlichen Daten des Kartenobjekts zurück.
\paragraph*{Beschreibung}
Diese Methode gibt die zusätzlichen Daten des Kartenobjekts zurück.
\paragraph*{Parameter}
\begin{itemize}
    \item keine
\end{itemize}
\paragraph*{Rückgabewert}
? zusätzlichen Daten des Kartenobjekts

\subsection{getOfficialName}%$$$M
\paragraph*{Kurzbeschreibung}
Diese Methode gibt den offiziellen Namen des Kartenobjekts zurück.
\paragraph*{Beschreibung}
Diese Methode gibt den offiziellen Namen des Kartenobjekts zurück.
\paragraph*{Parameter}
\begin{itemize}
    \item keine
\end{itemize}
\paragraph*{Rückgabewert}
String offizieller Name des Kartenobjekts

\subsection{getMapId}%$$$M
\paragraph*{Kurzbeschreibung}
Diese Methode gibt die KartenID des Kartenobjekts zurück.
\paragraph*{Beschreibung}
Diese Methode gibt die KartenID des Kartenobjekts zurück.
\paragraph*{Parameter}
\begin{itemize}
    \item keine
\end{itemize}
\paragraph*{Rückgabewert}
MapId KartenID des Kartenobjekts

\subsection{getGeocoding}%$$$M
\paragraph*{Beschreibung}
Diese Methode gibt die Geokodierung des Kartenobjekts zurück.
\paragraph*{Parameter}
\begin{itemize}
    \item keine
\end{itemize}
\paragraph*{Rückgabewert}
Geocoding Geokodierung


\section{Geocoding}
\paragraph*{Beschreibung}
Die Definition des Standorts.
\paragraph*{Anmerkung}
Die Geocodings werden lokal und auf dem Server gespeichert.