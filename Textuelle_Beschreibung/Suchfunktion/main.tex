\section{Search}
\paragraph*{Beschreibung}
Diese Klasse ermöglicht die Suche nach Gebäuden und Räumen über eine String-Eingabe in einem Textfeld, 
dies kann eine Gebäudenummer, Gebäudenummer und Raumnummer, ein Alias oder eine Person sein.
Zeigt eine Liste von Vorschlägen an, falls vorhanden, und ermöglicht es dem Benutzer, einen Vorschlag auszuwählen oder die Eingabe vollständig selbst vorzunehmen.
\paragraph*{Anmerkung}
Die Klasse wird nur lokal in der Anwendung verwendet, ohne jegliche Verbindung zum Server.

\subsection{getSearchInput}%$$$M
\paragraph*{Kurzbeschreibung}
Diese Methode ermöglicht Benutzereingaben und gibt einen gültigen Suchbegriffe oder Null zurück.
\paragraph*{Beschreibung}
Diese Methode ermöglicht die Suche von Gebäuden und Räumen über die Eingabe eines Suchbegriffs in ein Textfeld.
Zeigt eine Liste von Vorschlägen (die Suchbegriffe), falls vorhanden, und erlaubt dem Benutzer, einen Vorschlag auszuwählen oder die Eingabe vollständig selbst vorzunehmen.
\paragraph*{Parameter}
\begin{itemize}
    \item searchKeys : List<String> Liste aller Suchbegriffe
\end{itemize}
\paragraph*{Rückgabewert}
Suchbegriff, den der Benutzer eingegeben hat
\paragraph*{Notizen zur Implementierung}
Benutzt eine Instanz von SearchView (https://developer.android.com/guide/topics/search/search-dialog#SearchableActivity).

\subsection{getSearchable}%$$$M
\paragraph*{Kurzbeschreibung}
Diese Methode gibt das zum Suchbegriffe gehörenden Kartenobjekt zurück.
\paragraph*{Beschreibung}
Diese Klasse ermittelt das zu den Suchbegriffen gehörenden Kartenobjekt und gibt dieses zurück.
\paragraph*{Parameter}
\begin{itemize}
    \item searchKey : String Suchbegriff
\end{itemize}
\paragraph*{Rückgabewert}
Kartenobjekt


\section{Module: SearchView}
\paragraph*{Kurzbeschreibung}
Ein Widget, das dem Benutzer eine Benutzeroberfläche bietet, um eine Suchanfrage einzugeben und eine Anfrage an einen Suchanbieter zu senden. 
Zeigt eine Liste von Suchvorschlägen oder -ergebnissen an, falls verfügbar 
und ermöglicht es dem Benutzer, einen Vorschlag oder ein Ergebnis auszuwählen.
Es ist möglich letzte Suchen prorisiert vorzuschlagen, diese Funktion wird auch verwendet.
